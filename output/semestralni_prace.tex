% VYSOKOŠKOLSKÁ SEMESTRÁLNÍ PRÁCE
% autor: Lukáš Milar, Tomáš Prudký
% název: Zpracování dat pro předmět NMAST
% !TEX encoding = UTF-8 Unicode
\documentclass[a4paper, 12pt]{article}

\usepackage{cmap}
\usepackage[utf8]{inputenc}
\usepackage[T1]{fontenc}
\usepackage[czech]{babel}
\selectlanguage{czech}
\usepackage{vskpupa}
\usepackage{subfig}
\usepackage{longtable}

%%%%%%%%%%%%%%%%%%%%%%%%%%%%%%%%%%%%%%%%%%%%%%%%%%%%%%%%%%%%
% Údaje o práci
%%%%%%%%%%%%%%%%%%%%%%%%%%%%%%%%%%%%%%%%%%%%%%%%%%%%%%%%%%%%

\def\jmenoFakulty{Fakulta elektrotechniky a informatiky}
\def\jmenoAutora{Bc. Lukáš Milar, Bc. Tomáš Prudký}
\def\nazevPrace{Zpracování dat pro předmět NMAST}
\def\typPrace{Semestrální práce}
\def\rok{2021}

%%%%%%%%%%%%%%%%%%%%%%%%%%%%%%%%%%%%%%%%%%%%%%%%%%%%%%%%%%%%
% Začátek dokumentu
%%%%%%%%%%%%%%%%%%%%%%%%%%%%%%%%%%%%%%%%%%%%%%%%%%%%%%%%%%%%

\usepackage{Sweave}
\begin{document}
\Sconcordance{concordance:semestralni_prace.tex:semestralni_prace.Rnw:%
1 27 1 1 0 20 1 1 17 3 1 1 11 11 1 1 96 20 0 1 2 7 1 1 5 2 1 1 5 6 1 1 %
5 2 1 1 5 6 1 1 10 2 1 1 11 6 1 1 20 7 1 1 5 2 1 1 5 6 1 1 5 2 1 1 5 6 %
1 1 5 2 1 1 5 6 1 1 5 2 1 1 5 6 1 1 5 2 1 1 5 7 1 1 5 2 1 1 5 6 1 1 5 7 %
1 1 5 2 1 1 5 5 1 1 5 2 1 1 5 5 1 1 5 8 1 1 7 9 1 1 23 20 0 1 2 8 1 1 5 %
2 1 1 5 8 1 1 5 1 2 13 0 2 2 13 0 2 2 13 0 1 2 2 1 1 2 13 0 2 2 13 0 2 %
2 13 0 1 2 3 1 1 2 13 0 2 2 13 0 2 2 13 0 1 2 2 1 1 2 13 0 2 2 13 0 1 2 %
2 1 1 2 8 0 2 2 8 0 1 2 4 1 1 2 13 0 1 2 2 1 1 2 7 0 2 2 7 0 1 2 3 1 1 %
3 10 0 1 2 3 1 1 5 7 1 1 5 2 1 1 5 7 1 1 5 2 1 1 5 7 1 1 5 7 1 1 3 5 0 %
1 2 5 1 1 6 127 0 1 1 43 0 1 2 52 0 1 2 4 1 1 5 8 1 1 5 44 0 1 1 1 2 4 %
1 1 5 2 1 1 5 5 1 1 5 8 1 1 2 7 0 1 2 5 1 1 21 52 1}


%%%%%%%%%%%%%%%%%%%%%%%%%%%%%%%%%%%%%%%%%%%%%%%%%%%%%%%%%%%%
% Úvodní strany
%%%%%%%%%%%%%%%%%%%%%%%%%%%%%%%%%%%%%%%%%%%%%%%%%%%%%%%%%%%%

\titulniStrana
\generujObsah			% obsah
\generujSeznamObrazku		% seznam obrázků
\generujSeznamTabulek		% seznam tabulek

%%%%%%%%%%%%%%%%%%%%%%%%%%%%%%%%%%%%%%%%%%%%%%%%%%%%%%%%%%%%
% Úvod
%%%%%%%%%%%%%%%%%%%%%%%%%%%%%%%%%%%%%%%%%%%%%%%%%%%%%%%%%%%%


\clearpage
\pagestyle{plain}		% zapne číslování stránek (sazba zápatí)
\phantomsection \addcontentsline{toc}{section}{Úvod}
\section*{Úvod}
\label{uvod}

Tato semestrální práce se zabývá analýzou vývoje epidemie nemoci Covid-19 v ČR. Za tímto účelem
jsou srovnány přírůstky nových případů s našimi sousedy, efektivita testů při odhalování nových případů, úmrtnost nakažených, střední hodnota hospitalizovaných a vývoj střední hodnoty nových případů v čase. Dále je provedeno srovnání středních hodnot nových případů na milión s našimi sousedy a zkoumáno jaké rozdělení pravděpodobnosti data následují. Nakonec je pomocí regrese analyzováno
podle jaké funkce se řídí přírůstek nových padientů na ICU v závislosti na nových hospitalizacích. Použitá data čerpají ze zdroje~\cite{1}.

%% Načítání dat start



%% Načítání dat end

%%%%%%%%%%%%%%%%%%%%%%%%%%%%%%%%%%%%%%%%%%%%%%%%%%%%%%%%%%%%
% Kapitoly
%%%%%%%%%%%%%%%%%%%%%%%%%%%%%%%%%%%%%%%%%%%%%%%%%%%%%%%%%%%%

%% Popis dat
\section{Popis dat}
Data použitá v této práci se zabývají veličinami ohledně nemoci Covid-19 a pochází
od společnosti Our World in Data. Tato data jsou denně aktualizována a obsahují
například informace o očkování, testech, hospitalizacích, nových případech,
nových úmrtích či reprodukčním čísle. Veškeré hodnoty jsou pozorovány napříč
mnoha státy. Pro bližší popis těchto dat vizte zdroj~\cite{1}.

Covid-19 (též COVID-19; z anglického spojení coronavirus disease 2019, což česky znamená koronavirové onemocnění 2019; výslovnost: [kovid devatenáct]; podle ICD-11 označené XN109) je vysoce infekční onemocnění, které je způsobeno koronavirem SARS-CoV-2. První případ byl identifikován v čínském Wu-chanu v prosinci 2019. Od té doby se virus rozšířil po celém světě, což způsobilo přetrvávající pandemii.

Příznaky nemoci covid-19 jsou různé, od bezpříznakového stavu až po závažné onemocnění, ale často zahrnují horečku, kašel, únavu, dýchací potíže a ztrátu čichu a chuti. Příznaky začínají jeden až čtrnáct dní po vystavení viru. U přibližně jednoho z pěti infikovaných jedinců se neobjeví žádné příznaky. Zatímco většina lidí má mírné příznaky, u některých lidí se vyvine syndrom akutní dechové tísně. Tento syndrom může být přivoděn cytokinovými bouřemi, víceorgánovým selháním, septickým šokem a krevními sraženinami. Bylo pozorováno dlouhodobější poškození orgánů (zejména plic a srdce). Existuje obava z významného počtu pacientů, kteří se zotavili z akutní fáze onemocnění, ale nadále pociťují řadu následků – známých jako dlouhodobý covid-19 – i několik měsíců poté. Mezi tyto účinky patří silná únava, ztráta paměti a další kognitivní problémy, slabá horečka, svalová slabost a dušnost.

Virus, který způsobuje covid-19, se šíří hlavně vzdušným přenosem, když je infikovaná osoba v blízkém kontaktu s jinou osobou. Malé kapičky a aerosoly obsahující virus se mohou šířit z nosu a úst infikované osoby při dýchání, kašlání, kýchání, zpěvu nebo mluvení. Ostatní lidé se mohou nakazit, pokud se virus dostane do jejich úst, nosu nebo očí. Virus se může šířit také kontaminovaným povrchem, i když to není považováno za hlavní cestu přenosu. Přesná cesta přenosu je zřídkakdy přesvědčivě prokázána, ale k infekci dochází hlavně tehdy, když jsou lidé dostatečně blízko sebe. Virus se může šířit až dva dny předtím, než infikované osoby projeví příznaky, a od jedinců, kteří nikdy nepociťují příznaky. Lidé zůstávají infekční po dobu až deseti dnů při středně závažných případech a dva týdny ve vážných případech. Virus se šíří snadněji ve vnitřních prostorách a v davu. Pro diagnózu onemocnění byly vyvinuty různé testovací metody. Standardní diagnostickou metodou je reverzní transkripční polymerázová řetězová reakce v reálném čase (PCR test) výtěrem z nosohltanu.

Preventivní opatření zahrnují fyzický či společenský odstup, umístění ohrožených osob do karantény, větrání vnitřních prostor, zakrývání úst a nosu při kašli a kýchání, mytí rukou a udržování neumytých rukou pryč od obličeje. Aby se minimalizovalo riziko přenosu, bylo na veřejnosti doporučeno použití roušek, obličejových masek nebo jiného zakrytí dýchacích cest. Bylo vyvinuto několik vakcín proti covidu-19, načež většina států světa zahájila očkovací kampaně a samotné očkování, jehož rozsah je ovšem závislý na přístupnosti dostatečného množství vakcín.

Ačkoli probíhají práce na vývoji léků, které zpomalují a zastavují virus, primární léčba je v současnosti symptomatická. Zahrnuje léčbu příznaků, podpůrnou péči, izolaci a některá experimentální opatření.

Vzhledem k velkému množství dat jsou v této práci použity zpravidla údaje pro
Českou republiku, ze kterých je dále využit užší výčet dostupných veličin.

%% Popisná statistika
\section{Popisná statistika}

V tabulkách níže jsou zobrazeny hodnoty popisné statistiky pro veličiny nových případů,
7denního klouzavého průměru nových případů, nových případů na milion, 7denního klouzavého průměru nových případů na milion,
hospitalizovaných pacientů a hospitalizovaných pacientů na milion v České republice. Hodnoty 7denního klouzavého průměru
lépe zachycují tyto veličiny v rámci dlouhodobých trendů, jelikož je eliminováno zkreslení v podobě menšího počtu uskutečněných
testů například během víkendů.

% latex table generated in R 4.1.1 by xtable 1.8-4 package
% Fri Dec 10 17:39:13 2021
\begin{table}[ht]
\centering
\begin{tabular}{rrr}
  \hline
 & Nové případy & 7denní klouzavý průměr nových případů \\ 
  \hline
průměr & 2940.58 & 2936.89 \\ 
  modus & 75.00 & 57.57 \\ 
  medián & 416.00 & 422.29 \\ 
  max & 17773.00 & 12954.86 \\ 
  min & -2214.00 & 2.71 \\ 
  šikmost & 1.55 & 1.16 \\ 
  špičatost & 1.38 & -0.07 \\ 
  odchylka & 4277.10 & 3876.20 \\ 
  variance & 18293577.15 & 15024928.55 \\ 
   \hline
\end{tabular}
\caption{Části popisné statistiky aplikované na data nových případů a jejich 7denního klouzavého průměru v ČR od 7. 3. 2020} 
\label{table:popisStat}
\end{table}% latex table generated in R 4.1.1 by xtable 1.8-4 package
% Fri Dec 10 17:39:13 2021
\begin{table}[ht]
\centering
\begin{tabular}{rrr}
  \hline
 & Nové případy na milión & 7denní klouzavý průměr nových případů na milión \\ 
  \hline
průměr & 274.19 & 273.85 \\ 
  modus & 6.99 & 5.37 \\ 
  medián & 38.79 & 39.38 \\ 
  max & 1657.22 & 1207.96 \\ 
  min & -206.44 & 0.25 \\ 
  šikmost & 1.55 & 1.16 \\ 
  špičatost & 1.38 & -0.07 \\ 
  odchylka & 398.81 & 361.43 \\ 
  variance & 159052.40 & 130633.34 \\ 
   \hline
\end{tabular}
\caption{Části popisné statistiky aplikované na data nových případů na milión a jejich 7denního klouzavého průměru v ČR od 7. 3. 2020} 
\label{table:popisStat}
\end{table}% latex table generated in R 4.1.1 by xtable 1.8-4 package
% Fri Dec 10 17:39:13 2021
\begin{table}[ht]
\centering
\begin{tabular}{rrr}
  \hline
 & Hospitalizovaní pacienti & Hospitalizovaní pacienti na milión \\ 
  \hline
průměr & 2370.49 & 221.03 \\ 
  modus & 69.00 & 6.43 \\ 
  medián & 339.00 & 31.61 \\ 
  max & 9509.00 & 886.66 \\ 
  min & 0.00 & 0.00 \\ 
  šikmost & 0.86 & 0.86 \\ 
  špičatost & -0.86 & -0.86 \\ 
  odchylka & 2973.01 & 277.22 \\ 
  variance & 8838793.84 & 76848.36 \\ 
   \hline
\end{tabular}
\caption{Části popisné statistiky aplikované na data nových hospitalizací a nových hospitalizací na milión v ČR od 7. 3. 2020} 
\label{table:popisStat}
\end{table}


%% Základní grafy
\section{Základní grafy}
\subsection{Histogram}

Následující histogram zobrazuje četnost hodnot klouzavého průměru nových případů v ČR od 7. 3. 2020.
Vzhledem k očividnému zešikmení dat vlevo byla pro lepší přehlednost data zlogaritmována.

%% histogram start

\begin{figure}[H]
\centering
\subfloat[\centering Klouzavý průměr nových případů]
{\includegraphics[width=7cm]{histogram_new_cases_smoothed.pdf}}
\qquad
\subfloat[\centering Logaritmická transformace]
{\includegraphics[width=7cm]{histogram_new_cases_smoothed_log.pdf}}
\caption{Klouzavý průměr nových případů v ČR od 7. 3. 2020}
\end{figure}

\clearpage

Následující histogram zobrazuje četnost hodnot nových případů na milión obyvatel v ČR od 7. 3. 2020.
Vzhledem k očividnému zešikmení dat vlevo byla pro lepší přehlednost data opět zlogaritmována.

%% histogram start

\begin{figure}[H]
\centering
\subfloat[\centering Nové případy na milión]
{\includegraphics[width=7cm]{histogram_new_cases_per_million.pdf}}
\qquad
\subfloat[\centering Logaritmická transformace]
{\includegraphics[width=7cm]{histogram_new_cases_per_million_log.pdf}}
\caption{Nové případy na milión v ČR od 7. 3. 2020}
\end{figure}


Následující histogram zobrazuje četnost hodnot 7denního klouzavého průměru nových případů na milión obyvatel v ČR od 7. 3. 2020.
Vzhledem k očividnému zešikmení dat vlevo byla pro lepší přehlednost data opět zlogaritmována.

%% histogram start

\begin{figure}[H]
\centering
\subfloat[\centering Klouzavý průměr nových případů na milión]
{\includegraphics[width=7cm]{histogram_new_cases_smoothed_per_million.pdf}}
\qquad
\subfloat[\centering Logaritmická transformace]
{\includegraphics[width=7cm]{histogram_new_cases_smoothed_per_million_log.pdf}}
\caption{Klouzavý průměr nových případů na milión v ČR od 7. 3. 2020}
\end{figure}


Následující histogram zobrazuje četnost hodnot hospitalizovaných pacientů v ČR od 7. 3. 2020.
Vzhledem k očividnému zešikmení dat vlevo byla pro lepší přehlednost data opět zlogaritmována.


%% histogram start

\begin{figure}[H]
\centering
\subfloat[\centering Hospitalizovaní pacienti]
{\includegraphics[width=7cm]{histogram_hosp_patients.pdf}}
\qquad
\subfloat[\centering Logaritmická transformace]
{\includegraphics[width=7cm]{histogram_hosp_patients_log.pdf}}
\caption{Hospitalizovaní pacienti v ČR od 7. 3. 2020}
\end{figure}

Následující histogram zobrazuje četnost hodnot nově testovaných v ČR od 7. 3. 2020.
Vzhledem k očividnému zešikmení dat vlevo byla pro lepší přehlednost data opět zlogaritmována.

%% histogram start

\begin{figure}[H]
\centering
\subfloat[\centering Nově testovaní]
{\includegraphics[width=7cm]{histogram_new_tests.pdf}}
\qquad
\subfloat[\centering Logaritmická transformace]
{\includegraphics[width=7cm]{histogram_new_tests_log.pdf}}
\caption{Nově testovaní v ČR od 7. 3. 2020}
\end{figure}


Následující histogram zobrazuje četnost hodnot nových případů v ČR od 7. 3. 2020.
Vzhledem k očividnému zešikmení dat vlevo byla pro lepší přehlednost data opět zlogaritmována.

%% histogram start

\begin{figure}[H]
\centering
\subfloat[\centering Nové případy]
{\includegraphics[width=7cm]{histogram_new_cases.pdf}}
\qquad
\subfloat[\centering Logaritmická transformace]
{\includegraphics[width=7cm]{histogram_new_cases_log.pdf}}
\caption{Nové případy v ČR od 7. 3. 2020}
\end{figure}



Následující histogram zobrazuje srovnání četnosti hodnot klouzavého průměru nových případů v ČR a Rakousku od 7. 3. 2020.
Vzhledem k očividnému zešikmení dat vlevo byla pro lepší přehlednost data opět zlogaritmována.

%% histogram start

\begin{figure}[H]
\centering
\subfloat[\centering Nové případy na milión pro ČR a Rakousko]
{\includegraphics[width=7cm]{histogram_new_cases_cz_au}}
\qquad
\subfloat[\centering Logaritmická transformace]
{\includegraphics[width=7cm]{histogram_new_cases_cz_au_log.pdf}}
\caption{Klouzavý průměr nových případů na milión pro Česko a Rakousko od 7. 3. 2020}
\end{figure}


\subsection{Bodový graf}
\begin{figure}[H]
\centering

\includegraphics[width=10cm]{bodovy_nove_pripady_log.pdf}
\caption{Bodový graf zlogaritmovaných nových případů}

\includegraphics[width=10cm]{bodovy_nove_testy.pdf}
\caption{Bodový graf nových testů}

\end{figure}
\begin{figure}[H]
\centering

\includegraphics[width=10cm]{bodovy_reprodukcni_cislo.pdf}
\caption{Bodový graf reprodukčního čísla}

\includegraphics[width=10cm]{bodovy_icu_patients.pdf}
\caption{Bodový graf pacientů na icu}

\end{figure}
\begin{figure}[H]
\centering

\includegraphics[width=10cm]{bodovy_hosp_patients.pdf}
\caption{Bodový graf hospitalizovaných pacientů}

\includegraphics[width=10cm]{bodovy_weekly_icu_admissions.pdf}
\caption{Bodový graf týdenních přírůstků na icu}

\end{figure}
\begin{figure}[H]
\centering

\includegraphics[width=10cm]{bodovy_weekly_hosp_admissions.pdf}
\caption{Bodový graf týdenních hospitalizací}

\includegraphics[width=10cm]{bodovy_positive_rate.pdf}
\caption{Bodový graf pozitivity testů}

\end{figure}
\begin{figure}[H]
\centering

\includegraphics[width=10cm]{bodovy_new_vaccinations.pdf}
\caption{Bodový graf nových očkování}

\includegraphics[width=10cm]{bodovy_excess_mortality.pdf}
\caption{Bodový graf smrtnosti}

\end{figure}
\subsection{Boxplot}
\begin{figure}[H]
\centering

\includegraphics[width=10cm]{boxplot_new_cases_per_million.pdf}
\caption{Boxplot graf pro nové případy na milión}

\includegraphics[width=10cm]{boxplot_reproduction_rate.pdf}
\caption{Boxplot graf pro reprodukční číslo}
\end{figure}

\begin{figure}[H]
\centering

\includegraphics[width=10cm]{boxplot_log_new_deaths.pdf}
\caption{Boxplot graf pro zlogaritmované nové smrti}
\end{figure}

\subsection{3D graf}
\begin{figure}[H]
\centering

\includegraphics[width=10cm]{3D_graf_new_cases_new_tests_total_cases.pdf}
\caption{3D graf počtu případů a počtu testů}

\includegraphics[width=10cm]{3D_graf_log_new_cases_new_tests_total_cases.pdf}
\caption{3D graf zlogaritmovaných počtu případů a počtu testů}

\end{figure}
\begin{figure}[H]

\includegraphics[width=10cm]{3D_graf_new_cases_new_tests_new_vaccinations.pdf}
\caption{3D graf počtu případů a počtu nových očkování}

\includegraphics[width=10cm]{3D_graf_new_cases.pdf}
\caption{3D graf počtu nových případů}

\end{figure}
\begin{figure}[H]

\includegraphics[width=10cm]{3D_graf_reproduction_rate.pdf}
\caption{3D graf reprodukčního čísla}

\end{figure}

\subsection{Hexbin}
\begin{figure}[H]
\centering

\includegraphics[width=10cm]{hexbin_graf_log_new_cases_new_deaths.pdf}
\caption{Hexbin graf nových zlog. nových případů a nových úmrtí}

\end{figure}
\clearpage

\subsection{Chernoff faces}

Pomocí chernoff faces jsou vyobrazeny obličeje na základě hodnot pro Česko. Hodnoty, které jsou použity jsou
$new_cases$, $new_deaths$, $new_tests$, $new_vaccinations$, $icu_patiens$. Každý obličej reprezentuje jednu funkci, která
je aplikována na zvolené hodnoty. Z grafu je zřejmé zešikmení dat vzhledem k velkým rozdílům mezi hodnotami průměru
a mediánu zobrazovaných veličin.

% latex table generated in R 4.1.1 by xtable 1.8-4 package
% Fri Dec 10 17:39:14 2021
\begin{table}[ht]
\centering
\begin{tabular}{rrrrrr}
  \hline
 & new\_cases & new\_deaths & new\_tests & new\_vaccinations & icu\_patients \\ 
  \hline
prumer & 2940.58 & 54.33 & 94464.09 & 42220.25 & 424.51 \\ 
  sd & 4277.10 & 72.32 & 83099.10 & 35340.13 & 558.31 \\ 
  maximum & 17773.00 & 295.00 & 416333.00 & 121742.00 & 2062.00 \\ 
  minumum & -2214.00 & -6.00 & 4537.00 & 266.00 & 0.00 \\ 
  sikmost & 1.55 & 1.11 & 1.39 & 0.79 & 1.23 \\ 
  spicatost & 1.38 & -0.06 & 1.22 & -0.57 & 0.43 \\ 
  iqr & 4275.75 & 107.00 & 82303.50 & 49771.00 & 799.00 \\ 
  median & 416.00 & 7.00 & 65211.00 & 33315.00 & 74.00 \\ 
   \hline
\end{tabular}
\caption{Hodnoty dat znázorněných pomocí Chernoffových obličejů} 
\label{table:popisStat}
\end{table}
\begin{figure}[H]
\small
\centering

\begin{Schunk}
\begin{Soutput}
effect of variables:
 modified item       Var               
 "height of face   " "new_cases"       
 "width of face    " "new_deaths"      
 "structure of face" "new_tests"       
 "height of mouth  " "new_vaccinations"
 "width of mouth   " "icu_patients"    
 "smiling          " "new_cases"       
 "height of eyes   " "new_deaths"      
 "width of eyes    " "new_tests"       
 "height of hair   " "new_vaccinations"
 "width of hair   "  "icu_patients"    
 "style of hair   "  "new_cases"       
 "height of nose  "  "new_deaths"      
 "width of nose   "  "new_tests"       
 "width of ear    "  "new_vaccinations"
 "height of ear   "  "icu_patients"    
\end{Soutput}
\end{Schunk}
\caption{Legenda Chernoff faces grafu tabulky popisné statistiky}
\end{figure}

\begin{figure}[H]
\small
\centering

\includegraphics[width=10cm]{faces_graf_covid.pdf}
\caption{Chernoff faces graf tabulky popisné statistiky}
\end{figure}
\clearpage

\subsection{QQPlot}

Na základě následujícího QQPlot grafu je možné dojít k závěru, že počty nových
úmrtí a počty nových případů v ČR se řídí dle podobného rozdělení pravděpodobnosti.

\begin{figure}[H]
\centering

\includegraphics[width=10cm]{qqplot_graph_new_cases_new_deaths.pdf}
\caption{QQPlot graf nových případů a nových úmrtí}
\end{figure}
\clearpage

Na základě následujícího QQPlot grafu je možné dojít k závěru, že počty nových
případů a počty nových testů v ČR se řídí dle podobného rozdělení pravděpodobnosti.

\begin{figure}[H]
\centering

\includegraphics[width=10cm]{qqplot_graph_new_tests_new_cases.pdf}
\caption{QQPlot graf nových testů a nových případů}

\end{figure}
\clearpage

%% Testování statistických hypotéz
\section{Testování statistických hypotéz}

\subsection{Jednovýběrový Studentův test vůči střední hodnotě}
Následující test testuje zda se střední hodnota nových případů v ČR rovná
hodnotě 3300 s hladinou významnosti $\alpha$ = 0.05. Testová statistika nabývá hodnoty -2.0168 při 575 stupních volnosti. Vzhledem k tomu, že hodnota p-value je nižší než hladina významnosti, tuto hypotézu zamítáme ve prospěch hypotézy alternativní, tudíž že se střední hodnota
nových případů v ČR nerovná hodnotě 3300.

\begin{Schunk}
\begin{Soutput}
	One Sample t-test

data:  new_cases_czechia
t = -2.0168, df = 575, p-value = 0.04418
alternative hypothesis: true mean is not equal to 3300
95 percent confidence interval:
 2590.550 3290.603
sample estimates:
mean of x 
 2940.576 
\end{Soutput}
\end{Schunk}

\clearpage

Následující test testuje zda se střední hodnota 7denního klouzavého průměru nových případů v ČR rovná
hodnotě 3300 s hladinou významnosti $\alpha$ = 0.05. Testová statistika nabývá hodnoty -2.2482 při 575 stupních volnosti. Vzhledem k tomu, že hodnota p-value je nižší než hladina významnosti, tuto hypotézu zamítáme ve prospěch hypotézy alternativní, tudíž že se střední hodnota
nových případů v ČR nerovná hodnotě 3300.
\begin{Schunk}
\begin{Soutput}
	One Sample t-test

data:  new_cases_smoothed_czechia
t = -2.2482, df = 575, p-value = 0.02494
alternative hypothesis: true mean is not equal to 3300
95 percent confidence interval:
 2619.672 3254.109
sample estimates:
mean of x 
  2936.89 
\end{Soutput}
\end{Schunk}

Následující test testuje zda se střední hodnota nových případů na milión v ČR rovná
hodnotě 300 s hladinou významnosti $\alpha$ = 0.05. Testová statistika nabývá hodnoty -1.5531 při 575 stupních volnosti. Vzhledem k tomu, že hodnota p-value je vyšší než hladina významnosti, tuto hypotézu nemůžeme zamítnout ve prospěch hypotézy alternativní.
\begin{Schunk}
\begin{Soutput}
	One Sample t-test

data:  new_cases_per_million_czechia
t = -1.5531, df = 575, p-value = 0.1209
alternative hypothesis: true mean is not equal to 300
95 percent confidence interval:
 241.5531 306.8289
sample estimates:
mean of x 
  274.191 
\end{Soutput}
\end{Schunk}

\clearpage

Následující test testuje zda se střední hodnota 7denního klouzavého průměru nových případů na milión v ČR rovná
hodnotě 300 s hladinou významnosti $\alpha$ = 0.05. Testová statistika nabývá hodnoty -1.7366 při 575 stupních volnosti. Vzhledem k tomu, že hodnota p-value je vyšší než hladina významnosti, tuto hypotézu nemůžeme zamítnout ve prospěch hypotézy alternativní.
\begin{Schunk}
\begin{Soutput}
	One Sample t-test

data:  new_cases_smoothed_per_million_czechia
t = -1.7366, df = 575, p-value = 0.08299
alternative hypothesis: true mean is not equal to 300
95 percent confidence interval:
 244.2687 303.4260
sample estimates:
mean of x 
 273.8474 
\end{Soutput}
\end{Schunk}


Následující test testuje zda se střední hodnota nových hospitalizovaných pacientů v ČR rovná
hodnotě 2000 s hladinou významnosti $\alpha$ = 0.05. Testová statistika nabývá hodnoty 2.9726 při 568 stupních volnosti. Vzhledem k tomu, že hodnota p-value je nižší než hladina významnosti, tuto hypotézu zamítáme ve prospěch hypotézy alternativní, tudíž že se střední hodnota
nových hospitalizovaných pacientů v ČR nerovná hodnotě 2000.
\begin{Schunk}
\begin{Soutput}
	One Sample t-test

data:  hosp_patients_czechia
t = 2.9726, df = 568, p-value = 0.003078
alternative hypothesis: true mean is not equal to 2000
95 percent confidence interval:
 2125.690 2615.294
sample estimates:
mean of x 
 2370.492 
\end{Soutput}
\end{Schunk}

\clearpage

Následující test testuje zda se střední hodnota nových hospitalizovaných pacientů na milión v ČR rovná
hodnotě 200 s hladinou významnosti $\alpha$ = 0.05. Testová statistika nabývá hodnoty 1.8099 při 568 stupních volnosti. Vzhledem k tomu, že hodnota p-value je vyšší než hladina významnosti, tuto hypotézu nemůžeme zamítnout ve prospěch hypotézy alternativní.
\begin{Schunk}
\begin{Soutput}
	One Sample t-test

data:  hosp_patients_per_million_czechia
t = 1.8099, df = 568, p-value = 0.07083
alternative hypothesis: true mean is not equal to 200
95 percent confidence interval:
 198.2078 243.8604
sample estimates:
mean of x 
 221.0341 
\end{Soutput}
\end{Schunk}

\clearpage

\subsection{Dvouvýběrový Studentův test}

Následující dvouvýběrový t-test testuje hypotézu, že střední hodnota nových případů v
první části dat z ČR je rovna střední hodnotě v druhé části. Vzhledem ke skutečnosti, že p-value je menší
než hladina významnosti ($\alpha$ = 0,05), zamítáme tuto hypotézu ve prospěch alternativní.
Při této hladině významnosti tudíž můžeme tvrdit, že střední hodnota nových případů v první
části dat z ČR se nerovná střední hodnotě z druhé části.

\begin{Schunk}
\begin{Soutput}
	Welch Two Sample t-test

data:  new_cases_czechia_p1 and new_cases_czechia_p2
t = -4.518, df = 537.03, p-value = 7.683e-06
alternative hypothesis: true difference in means is not equal to 0
95 percent confidence interval:
 -2272.4359  -895.1752
sample estimates:
mean of x mean of y 
 2148.674  3732.479 
\end{Soutput}
\end{Schunk}

\clearpage

Následující dvouvýběrový t-test testuje hypotézu, že střední hodnota nových případů
v ČR je rovna střední hodnotě nových případů v Německu. Testová statistika nabývá hodnoty
-11,133 při 843,1 stupních volnosti. Vzhledem ke skutečnosti, že p-value je menší
než hladina významnosti ($\alpha$ = 0,05), zamítáme tuto hypotézu ve prospěch alternativní.
Při této hladině významnosti tudíž můžeme tvrdit, že střední hodnota nových případů v Německu
se nerovná střední hodnotě nových případů v ČR.
\begin{Schunk}
\begin{Soutput}
	Welch Two Sample t-test

data:  new_cases_czechia and new_cases_germany
t = -11.133, df = 843.1, p-value < 2.2e-16
alternative hypothesis: true difference in means is not equal to 0
95 percent confidence interval:
 -5240.394 -3669.509
sample estimates:
mean of x mean of y 
 2940.576  7395.528 
\end{Soutput}
\end{Schunk}

Následující dvouvýběrový t-test testuje hypotézu, že střední hodnota nových případů na milión
v ČR je rovna střední hodnotě nových případů na milión na Slovensku. Vzhledem ke skutečnosti, že p-value je menší
než hladina významnosti ($\alpha$ = 0,05), zamítáme tuto hypotézu ve prospěch alternativní.
Při této hladině významnosti tudíž můžeme tvrdit, že střední hodnota nových případů na milión na Slovensku
se nerovná střední hodnotě nových případů na milión v ČR.

\begin{Schunk}
\begin{Soutput}
	Welch Two Sample t-test

data:  new_cases_per_million_czechia and new_cases_per_million_slovakia
t = 7.7283, df = 817.84, p-value = 3.194e-14
alternative hypothesis: true difference in means is not equal to 0
95 percent confidence interval:
 105.8863 177.9857
sample estimates:
mean of x mean of y 
  274.191   132.255 
\end{Soutput}
\end{Schunk}

\clearpage

Následující dvouvýběrový t-test testuje hypotézu, že střední hodnota nových případů na milión
v ČR je rovna střední hodnotě nových případů na milión v Německu. Vzhledem ke skutečnosti, že p-value je menší
než hladina významnosti ($\alpha$ = 0,05), zamítáme tuto hypotézu ve prospěch alternativní.
Při této hladině významnosti tudíž můžeme tvrdit, že střední hodnota nových případů na milión v Německu
se nerovná střední hodnotě nových případů na milión v ČR.

\begin{Schunk}
\begin{Soutput}
	Two Sample t-test

data:  new_cases_per_million_czechia and new_cases_per_million_germany
t = 10.844, df = 1150, p-value < 2.2e-16
alternative hypothesis: true difference in means is not equal to 0
95 percent confidence interval:
 152.3817 219.7075
sample estimates:
mean of x mean of y 
274.19103  88.14644 
\end{Soutput}
\end{Schunk}

\clearpage

Následující dvouvýběrový t-test testuje hypotézu, že střední hodnota nových případů na milión
v ČR je rovna střední hodnotě nových případů na milión v Polsku. Vzhledem ke skutečnosti, že p-value je menší
než hladina významnosti ($\alpha$ = 0,05), zamítáme tuto hypotézu ve prospěch alternativní.
Při této hladině významnosti tudíž můžeme tvrdit, že střední hodnota nových případů na milión v Polsku
se nerovná střední hodnotě nových případů na milión v ČR.

\begin{Schunk}
\begin{Soutput}
	Two Sample t-test

data:  new_cases_per_million_czechia and new_cases_per_million_poland
t = 7.5404, df = 1150, p-value = 9.477e-14
alternative hypothesis: true difference in means is not equal to 0
95 percent confidence interval:
 103.9326 177.0431
sample estimates:
mean of x mean of y 
 274.1910  133.7032 
\end{Soutput}
\end{Schunk}

Následující dvouvýběrový t-test testuje hypotézu, že střední hodnota nových případů na milión
v ČR je rovna střední hodnotě nových případů na milión v Rakousku. Vzhledem ke skutečnosti, že p-value je menší
než hladina významnosti ($\alpha$ = 0,05), zamítáme tuto hypotézu ve prospěch alternativní.
Při této hladině významnosti tudíž můžeme tvrdit, že střední hodnota nových případů na milión v Rakousku
se nerovná střední hodnotě nových případů na milión v ČR.

\begin{Schunk}
\begin{Soutput}
	Two Sample t-test

data:  new_cases_per_million_czechia and new_cases_per_million_austria
t = 7.2242, df = 1150, p-value = 9.157e-13
alternative hypothesis: true difference in means is not equal to 0
95 percent confidence interval:
  95.01377 165.86690
sample estimates:
mean of x mean of y 
 274.1910  143.7507 
\end{Soutput}
\end{Schunk}

\subsection{Wilcox test}

Následující Wilcoxonův testuje hypotézu, že střední hodnota nových případů na milión
v ČR je rovna střední hodnotě nových případů na milión na Slovensku. Vzhledem ke skutečnosti, že p-value je menší
než hladina významnosti ($\alpha$ = 0,05), zamítáme tuto hypotézu ve prospěch alternativní.
Při této hladině významnosti tudíž můžeme tvrdit, že střední hodnota nových případů na milión na Slovensku
se nerovná střední hodnotě nových případů na milión v ČR. Vzhledem k zešikmení dat poskytuje tento test přesnější
výsledky oproti dvouvýběrovému t-testu.

\begin{Schunk}
\begin{Soutput}
	Wilcoxon rank sum test with continuity correction

data:  new_cases_per_million_czechia and new_cases_per_million_slovakia
W = 205293, p-value = 2.97e-12
alternative hypothesis: true location shift is not equal to 0
\end{Soutput}
\end{Schunk}

Následující Wilcoxonův testuje hypotézu, že střední hodnota nových případů na milión
v ČR je rovna střední hodnotě nových případů na milión v Německu. Vzhledem ke skutečnosti, že p-value je menší
než hladina významnosti ($\alpha$ = 0,05), zamítáme tuto hypotézu ve prospěch alternativní.
Při této hladině významnosti tudíž můžeme tvrdit, že střední hodnota nových případů na milión v Německu
se nerovná střední hodnotě nových případů na milión v ČR. Vzhledem k zešikmení dat poskytuje tento test přesnější
výsledky oproti dvouvýběrovému t-testu.

\begin{Schunk}
\begin{Soutput}
	Wilcoxon rank sum test with continuity correction

data:  new_cases_per_million_czechia and new_cases_per_million_germany
W = 188720, p-value = 5.261e-05
alternative hypothesis: true location shift is not equal to 0
\end{Soutput}
\end{Schunk}

\clearpage

\subsection{Fisherův test}

Následující Fisherův test zkoumá zda jsou rozptyly hodnot nových případů na milión
v ČR a na Slovensku stejné. Vzhledem ke skutečnosti, že p-value je menší
než hladina významnosti ($\alpha$ = 0,05), zamítáme tuto hypotézu ve prospěch alternativní,
tudíž že jsou rozptyly těchto dat různé.

\begin{Schunk}
\begin{Soutput}
	F test to compare two variances

data:  new_cases_per_million_czechia and new_cases_per_million_slovakia
F = 4.5141, num df = 575, denom df = 575, p-value < 2.2e-16
alternative hypothesis: true ratio of variances is not equal to 1
95 percent confidence interval:
 3.832723 5.316656
sample estimates:
ratio of variances 
          4.514119 
\end{Soutput}
\end{Schunk}

\subsection{Shapiro Wilk test}

Následující Shapiro Wilk test testuje zda je veličina nových případů v ČR nabývá
normálního rozdělení. Vzhledem ke skutečnosti, že p-value je menší
než hladina významnosti ($\alpha$ = 0,05), zamítáme tuto hypotézu ve prospěch alternativní,
tudíž že tato veličina nenabývá normálního rozdělení.

\begin{Schunk}
\begin{Soutput}
	Shapiro-Wilk normality test

data:  new_cases_czechia
W = 0.72003, p-value < 2.2e-16
\end{Soutput}
\end{Schunk}

\clearpage

Následující Shapiro Wilk test testuje zda je veličina nových testů v ČR nabývá
normálního rozdělení. Vzhledem ke skutečnosti, že p-value je menší
než hladina významnosti ($\alpha$ = 0,05), zamítáme tuto hypotézu ve prospěch alternativní,
tudíž že tato veličina nenabývá normálního rozdělení.

\begin{Schunk}
\begin{Soutput}
	Shapiro-Wilk normality test

data:  new_tests_czechia
W = 0.82915, p-value < 2.2e-16
\end{Soutput}
\end{Schunk}

%% ANOVA
\section{ANOVA}

\begin{Schunk}
\begin{Soutput}
                                Df   Sum Sq  Mean Sq F value   Pr(>F)    
new_cases_per_million_germany    1 32058325 32058325 1020.43  < 2e-16 ***
new_cases_per_million_slovakia   1 37842394 37842394 1204.54  < 2e-16 ***
new_cases_per_million_poland     1  2637470  2637470   83.95  < 2e-16 ***
new_cases_per_million_austria    1   978148   978148   31.14 3.72e-08 ***
Residuals                      571 17938795    31416                     
---
Signif. codes:  0 '***' 0.001 '**' 0.01 '*' 0.05 '.' 0.1 ' ' 1
\end{Soutput}
\end{Schunk}

\begin{figure}[H]
\centering

\includegraphics[width=10cm]{anova_graph_1.pdf}
\caption{Anova graf nových testů, případů a úmrtí}

\end{figure}

\begin{figure}[H]
\centering

\includegraphics[width=10cm]{anova_graph_2.pdf}
\caption{Anova graf nových testů, případů a úmrtí}

\includegraphics[width=10cm]{anova_graph_3.pdf}
\caption{Anova graf nových testů, případů a úmrtí}

\end{figure}

\begin{figure}[H]
\centering

\includegraphics[width=10cm]{anova_graph_4.pdf}
\caption{Anova graf nových testů, případů a úmrtí}

\includegraphics[width=10cm]{anova_graph_5.pdf}
\caption{Anova graf nových testů, případů a úmrtí}

\end{figure}

\begin{figure}[H]
\centering

\includegraphics[width=10cm]{anova_graph_6.pdf}
\caption{Anova graf nových testů, případů a úmrtí}

\end{figure}

\clearpage
%% Variance
\section{Variance}

Níže jsou popsány střední hodnoty kvadrátů odchylek od střední hodnoty nových testů
a nových případů v ČR.
\begin{Schunk}
\begin{Soutput}
     Min.   1st Qu.    Median      Mean   3rd Qu.      Max. 
-61749419 -61749419 -61749419 -61749419 -61749419 -61749419 
\end{Soutput}
\end{Schunk}

%% Korelace
\section{Korelace}
\subsection{Korelační matice}
\begin{table}[H]
\footnotesize
\begin{Schunk}
\begin{Soutput}
             [,1]        [,2]         [,3]         [,4]        [,5]        [,6]
 [1,]  1.00000000 -0.62274409 -0.129186615 -0.285154793  0.66579693  0.57478261
 [2,] -0.62274409  1.00000000  0.151838159  0.209449286 -0.43366681 -0.44705372
 [3,] -0.12918662  0.15183816  1.000000000  0.250435635 -0.36067001 -0.33492826
 [4,] -0.28515479  0.20944929  0.250435635  1.000000000 -0.10668415 -0.03613412
 [5,]  0.66579693 -0.43366681 -0.360670011 -0.106684148  1.00000000  0.50445750
 [6,]  0.57478261 -0.44705372 -0.334928261 -0.036134119  0.50445750  1.00000000
 [7,]  0.05132667 -0.14357190 -0.163185379  0.386541958  0.15705896 -0.13049153
 [8,]  0.89739130 -0.55838227  0.004784689 -0.171528226  0.57403785  0.69391304
 [9,]  0.30782609 -0.09088933 -0.482383691  0.227688483  0.39791260  0.43652174
[10,]  0.72869565 -0.63448577  0.187472832 -0.359599784  0.45966516  0.17739130
[11,] -0.66869565  0.54794522 -0.131361474  0.079233851 -0.46488368 -0.32956522
[12,]  0.42782609 -0.34007394  0.016093955  0.404005569  0.37442923  0.02869565
[13,]  0.38695652 -0.12002610 -0.172683792 -0.201567434  0.10784954  0.64347826
[14,] -0.27478261  0.13785606 -0.133971305  0.135394108 -0.09306371 -0.34956522
[15,]  0.61043478 -0.47358122  0.410613349 -0.280365934  0.17612525  0.16434783
[16,] -0.59304348  0.45749077 -0.321879108  0.026121050 -0.31919983 -0.23913043
[17,] -0.31652174  0.21743858  0.533710359  0.180235243 -0.17090672 -0.62434783
[18,] -0.59478261  0.47271147 -0.263157920  0.048323942 -0.32398348 -0.08000000
[19,] -0.60608696  0.46879758  0.355371934  0.154549544 -0.33529029 -0.68608696
[20,] -0.34398783  0.31165724  0.099412663 -0.006749405 -0.46889952  0.13089802
[21,]  0.32347826 -0.23265928 -0.586341943 -0.168916121  0.30876278  0.09043478
[22,]  0.23483367 -0.11048282  0.382423329 -0.190289684 -0.11439756  0.20656665
[23,]  0.25701240 -0.24880383 -0.511638037  0.139995729  0.47498913  0.25179387
[24,]  0.59143293 -0.52522836  0.291929526  0.050729401  0.30230535  0.09088933
             [,7]         [,8]          [,9]       [,10]        [,11]
 [1,]  0.05132667  0.897391304  0.3078260870  0.72869565 -0.668695652
 [2,] -0.14357190 -0.558382270 -0.0908893259 -0.63448577  0.547945218
 [3,] -0.16318538  0.004784689 -0.4823836907  0.18747283 -0.131361474
 [4,]  0.38654196 -0.171528226  0.2276884833 -0.35959978  0.079233851
 [5,]  0.15705896  0.574037848  0.3979125991  0.45966516 -0.464883681
 [6,] -0.13049153  0.693913043  0.4365217391  0.17739130 -0.329565217
 [7,]  1.00000000 -0.029578080  0.2801218186 -0.05480644 -0.334058318
 [8,] -0.02957808  1.000000000  0.3043478261  0.62956522 -0.664347826
 [9,]  0.28012182  0.304347826  1.0000000000 -0.23304348  0.008695652
[10,] -0.05480644  0.629565217 -0.2330434783  1.00000000 -0.646086957
[11,] -0.33405832 -0.664347826  0.0086956522 -0.64608696  1.000000000
[12,]  0.68595048  0.310434783  0.2269565217  0.28782609 -0.530434783
[13,] -0.48107878  0.443478261  0.2217391304  0.12000000  0.070434783
[14,]  0.38190521 -0.340869565  0.3634782609 -0.34260870  0.267826087
[15,] -0.33014357  0.658260870 -0.2886956522  0.82173913 -0.471304348
[16,] -0.17964334 -0.653043478  0.1313043478 -0.61739130  0.822608696
[17,]  0.07655503 -0.414782609 -0.6886956522  0.14695652 -0.069565217
[18,] -0.32709877 -0.603478261 -0.0008695652 -0.72869565  0.737391304
[19,]  0.27185735 -0.649565217 -0.5060869565 -0.27565217  0.147826087
[20,] -0.39373505 -0.147423356 -0.2065666498 -0.38617092  0.286584047
[21,]  0.39277951  0.199130435  0.4782608696  0.12782609  0.047826087
[22,] -0.58385905  0.359643409 -0.4431398235  0.50315287 -0.113937815
[23,]  0.34152709  0.133507287  0.7553816226 -0.08741031  0.153946514
[24,]  0.15749402  0.499238977  0.0239182437  0.65405524 -0.639704299
            [,12]        [,13]       [,14]       [,15]        [,16]
 [1,]  0.42782609  0.386956522 -0.27478261  0.61043478 -0.593043478
 [2,] -0.34007394 -0.120026095  0.13785606 -0.47358122  0.457490770
 [3,]  0.01609396 -0.172683792 -0.13397130  0.41061335 -0.321879108
 [4,]  0.40400557 -0.201567434  0.13539411 -0.28036593  0.026121050
 [5,]  0.37442923  0.107849535 -0.09306371  0.17612525 -0.319199834
 [6,]  0.02869565  0.643478261 -0.34956522  0.16434783 -0.239130435
 [7,]  0.68595048 -0.481078775  0.38190521 -0.33014357 -0.179643340
 [8,]  0.31043478  0.443478261 -0.34086957  0.65826087 -0.653043478
 [9,]  0.22695652  0.221739130  0.36347826 -0.28869565  0.131304348
[10,]  0.28782609  0.120000000 -0.34260870  0.82173913 -0.617391304
[11,] -0.53043478  0.070434783  0.26782609 -0.47130435  0.822608696
[12,]  1.00000000 -0.394782609  0.26521739  0.06782609 -0.467826087
[13,] -0.39478261  1.000000000 -0.48782609  0.35130435  0.007826087
[14,]  0.26521739 -0.487826087  1.00000000 -0.43130435  0.386086957
[15,]  0.06782609  0.351304348 -0.43130435  1.00000000 -0.601739130
[16,] -0.46782609  0.007826087  0.38608696 -0.60173913  1.000000000
[17,]  0.19130435 -0.588695652 -0.02869565  0.07826087 -0.173043478
[18,] -0.62347826  0.149565217 -0.10695652 -0.61739130  0.705217391
[19,]  0.05565217 -0.714782609  0.23130435 -0.38956522  0.089565217
[20,] -0.54185694  0.380082636 -0.40443576 -0.05044575 -0.036964558
[21,]  0.24869565  0.028695652  0.33217391 -0.09565217  0.119130435
[22,] -0.27788650  0.475320733 -0.51880845  0.73407264 -0.301369870
[23,]  0.26962384  0.076973257  0.49749947 -0.24309633  0.280930644
[24,]  0.51663406 -0.157860408 -0.22048272  0.50750164 -0.675799103
              [,17]         [,18]       [,19]         [,20]       [,21]
 [1,] -0.3165217391 -0.5947826087 -0.60608696 -0.3439878316  0.32347826
 [2,]  0.2174385787  0.4727114702  0.46879758  0.3116572423 -0.23265928
 [3,]  0.5337103593 -0.2631579196  0.35537193  0.0994126628 -0.58634194
 [4,]  0.1802352430  0.0483239420  0.15454954 -0.0067494053 -0.16891612
 [5,] -0.1709067229 -0.3239834823 -0.33529029 -0.4688995215  0.30876278
 [6,] -0.6243478261 -0.0800000000 -0.68608696  0.1308980244  0.09043478
 [7,]  0.0765550312 -0.3270987695  0.27185735 -0.3937350539  0.39277951
 [8,] -0.4147826087 -0.6034782609 -0.64956522 -0.1474233564  0.19913043
 [9,] -0.6886956522 -0.0008695652 -0.50608696 -0.2065666498  0.47826087
[10,]  0.1469565217 -0.7286956522 -0.27565217 -0.3861709158  0.12782609
[11,] -0.0695652174  0.7373913043  0.14782609  0.2865840468  0.04782609
[12,]  0.1913043478 -0.6234782609  0.05565217 -0.5418569382  0.24869565
[13,] -0.5886956522  0.1495652174 -0.71478261  0.3800826356  0.02869565
[14,] -0.0286956522 -0.1069565217  0.23130435 -0.4044357565  0.33217391
[15,]  0.0782608696 -0.6173913043 -0.38956522 -0.0504457503 -0.09565217
[16,] -0.1730434783  0.7052173913  0.08956522 -0.0369645584  0.11913043
[17,]  1.0000000000 -0.1191304348  0.69826087  0.0008697543 -0.46260870
[18,] -0.1191304348  1.0000000000  0.12608696  0.4322678945 -0.14434783
[19,]  0.6982608696  0.1260869565  1.00000000  0.0204392264 -0.33043478
[20,]  0.0008697543  0.4322678945  0.02043923  1.0000000000 -0.42574474
[21,] -0.4626086957 -0.1443478261 -0.33043478 -0.4257447372  1.00000000
[22,]  0.1021961320 -0.2152641930 -0.28614917  0.3166594171 -0.37660362
[23,] -0.4435747006 -0.0482713645 -0.38008264 -0.4354066986  0.50271799
[24,]  0.2083061584 -0.5953468286 -0.17308111 -0.2901261418 -0.11176343
           [,22]       [,23]       [,24]
 [1,]  0.2348337  0.25701240  0.59143293
 [2,] -0.1104828 -0.24880383 -0.52522836
 [3,]  0.3824233 -0.51163804  0.29192953
 [4,] -0.1902897  0.13999573  0.05072940
 [5,] -0.1143976  0.47498913  0.30230535
 [6,]  0.2065666  0.25179387  0.09088933
 [7,] -0.5838591  0.34152709  0.15749402
 [8,]  0.3596434  0.13350729  0.49923898
 [9,] -0.4431398  0.75538162  0.02391824
[10,]  0.5031529 -0.08741031  0.65405524
[11,] -0.1139378  0.15394651 -0.63970430
[12,] -0.2778865  0.26962384  0.51663406
[13,]  0.4753207  0.07697326 -0.15786041
[14,] -0.5188084  0.49749947 -0.22048272
[15,]  0.7340726 -0.24309633  0.50750164
[16,] -0.3013699  0.28093064 -0.67579910
[17,]  0.1021961 -0.44357470  0.20830616
[18,] -0.2152642 -0.04827136 -0.59534683
[19,] -0.2861492 -0.38008264 -0.17308111
[20,]  0.3166594 -0.43540670 -0.29012614
[21,] -0.3766036  0.50271799 -0.11176343
[22,]  1.0000000 -0.41626794  0.27533710
[23,] -0.4162679  1.00000000  0.03871248
[24,]  0.2753371  0.03871248  1.00000000
\end{Soutput}
\begin{Soutput}
       V1                V2                 V3                  V4          
 Min.   :-0.6687   Min.   :-0.63449   Min.   :-0.586342   Min.   :-0.35960  
 1st Qu.:-0.3234   1st Qu.:-0.43701   1st Qu.:-0.277838   1st Qu.:-0.16957  
 Median : 0.2459   Median :-0.11525   Median :-0.062201   Median : 0.04953  
 Mean   : 0.1093   Mean   :-0.04194   Mean   : 0.003863   Mean   : 0.06190  
 3rd Qu.: 0.5789   3rd Qu.: 0.24099   3rd Qu.: 0.260809   3rd Qu.: 0.18754  
 Max.   : 1.0000   Max.   : 1.00000   Max.   : 1.000000   Max.   : 1.00000  
       V5                 V6                 V7                 V8         
 Min.   :-0.46890   Min.   :-0.68609   Min.   :-0.58386   Min.   :-0.6643  
 1st Qu.:-0.32040   1st Qu.:-0.26174   1st Qu.:-0.21651   1st Qu.:-0.3593  
 Median : 0.13245   Median : 0.09066   Median : 0.01087   Median : 0.1663  
 Mean   : 0.09632   Mean   : 0.07237   Mean   : 0.04299   Mean   : 0.1031  
 3rd Qu.: 0.41335   3rd Qu.: 0.29798   3rd Qu.: 0.29547   3rd Qu.: 0.5179  
 Max.   : 1.00000   Max.   : 1.00000   Max.   : 1.00000   Max.   : 1.0000  
       V9                V10               V11                V12         
 Min.   :-0.68870   Min.   :-0.7287   Min.   :-0.66870   Min.   :-0.6235  
 1st Qu.:-0.21319   1st Qu.:-0.3469   1st Qu.:-0.46649   1st Qu.:-0.2934  
 Median : 0.17652   Median : 0.1239   Median :-0.03043   Median : 0.2091  
 Mean   : 0.09266   Mean   : 0.0616   Mean   :-0.03723   Mean   : 0.0917  
 3rd Qu.: 0.32174   3rd Qu.: 0.4705   3rd Qu.: 0.18242   3rd Qu.: 0.3264  
 Max.   : 1.00000   Max.   : 1.0000   Max.   : 1.00000   Max.   : 1.0000  
      V13                V14                V15                V16          
 Min.   :-0.71478   Min.   :-0.51881   Min.   :-0.61739   Min.   :-0.67580  
 1st Qu.:-0.17990   1st Qu.:-0.34130   1st Qu.:-0.34500   1st Qu.:-0.35837  
 Median : 0.07370   Median :-0.06088   Median : 0.00869   Median :-0.10500  
 Mean   : 0.04768   Mean   : 0.01106   Mean   : 0.05447   Mean   :-0.04807  
 3rd Qu.: 0.35850   3rd Qu.: 0.28391   3rd Qu.: 0.43484   3rd Qu.: 0.16871  
 Max.   : 1.00000   Max.   : 1.00000   Max.   : 1.00000   Max.   : 1.00000  
      V17                V18               V19                V20          
 Min.   :-0.68870   Min.   :-0.7287   Min.   :-0.71478   Min.   :-0.54186  
 1st Qu.:-0.34109   1st Qu.:-0.3940   1st Qu.:-0.38245   1st Qu.:-0.38806  
 Median :-0.01391   Median :-0.1130   Median :-0.07632   Median :-0.04371  
 Mean   :-0.02777   Mean   :-0.0717   Mean   :-0.07138   Mean   :-0.04832  
 3rd Qu.: 0.18300   3rd Qu.: 0.1320   3rd Qu.: 0.17374   3rd Qu.: 0.16982  
 Max.   : 1.00000   Max.   : 1.0000   Max.   : 1.00000   Max.   : 1.00000  
      V21                V22                V23                V24          
 Min.   :-0.58634   Min.   :-0.58386   Min.   :-0.51164   Min.   :-0.67580  
 1st Qu.:-0.18485   1st Qu.:-0.28995   1st Qu.:-0.24452   1st Qu.:-0.18493  
 Median : 0.06913   Median :-0.11221   Median : 0.13675   Median : 0.07081  
 Mean   : 0.05270   Mean   : 0.02674   Mean   : 0.09834   Mean   : 0.07580  
 3rd Qu.: 0.31244   3rd Qu.: 0.32741   3rd Qu.: 0.29608   3rd Qu.: 0.35154  
 Max.   : 1.00000   Max.   : 1.00000   Max.   : 1.00000   Max.   : 1.00000  
\end{Soutput}
\begin{Soutput}
      [,1] [,2] [,3] [,4] [,5] [,6] [,7] [,8] [,9] [,10] [,11] [,12] [,13]
 [1,]    1  307   86  111   31   75  294  239 1674  8617  9252  6305  5304
 [2,]   12  200   79   61   83   51  278  133 2136  4631 12089  4464  3383
 [3,]    0  350   52   74   53  119  203  258 3123  3105 15731  2665  7891
 [4,]   10  233   41   33   33  129  125  367 2107  4311 13229  1074 10898
 [5,]   50  381   59   59   40  105  101  403 2044  8326 11552  3575 14149
 [6,]    3  115   75   77   47   82  208  218  984  9543  7723  5176  4402
 [7,]   47  235  103   65   51  114  278  608 1474  9720  3608  4568  2706
 [8,]   48  195   55   47  118   59  243  317 2394 11102  2353  4621  3030
 [9,]   64  295   18   48  126   64  202  273 2306  8715 12699  4743  3741
[10,]   45  257   26   36   42  103  329  251 2910  5058  8925  1773 10862
[11,]   98  163   38   54   50  134  175  499 2944  8077  7870  2654 16420
[12,]   68   99   77   56   25  137  118  656 1976 11984  7355     0 17039
[13,]  230  160   78   34  127  130  141  679 1303 14969  4199  4237 13361
[14,]  139   68   57   38  127  113  289  797 1286 14150  1887  5857  3447
[15,]  162   52   46   34   93   90  292  503 1960 15258  5407  6406  5012
[16,]  125  105   18   62  168  153  326  404 2920 12474  4246  5872  6233
[17,]  116  217   28   74  260  226  292  560 3492  7300  5515  6217 12921
[18,]  158  116   53   56  305  246  198 1161 3796 10273  6471  3657 17332
[19,]  260   57   45   35  202  230  121 1159 2554 15663  5808  2000 17773
[20,]  271  140   48   38  149  281  190 1377 1841 12980  3187  5172 14861
[21,]  354  154   82   61   92  131  281 1447 3120 13055  1513  7897 13115
[22,]  352  133   55   69  132  112  315 1541 4456 13605  4375  8256  8449
[23,]  186   99   49   54  141  192  247  787 5338 11429  5896  7602  4289
[24,]  184   55   20   73  121  283  506 1034 5397  6542  2681  8830  9345
      [,14] [,15] [,16] [,17] [,18] [,19] [,20] [,21] [,22] [,23] [,24]
 [1,] 10801  8635 12191  7946   830  1280   154   172   200   221   421
 [2,] 10922  4863 16816  7714  3319  1231    90   148    69   299   451
 [3,]  8083  2430 15230  4008  3831  1060   391   173   138   212   253
 [4,]  9301  7727 14714  1743  2971   602   384   188   217   175   136
 [5,]  5253 10283 13162  6896  2629   259   267   139   253   174   487
 [6,]  2693  9537  9167  8664  2607  1120   190   105   202   134   555
 [7,]  7651  9014  3960  7285  1315  1195   183   130   210   107   524
 [8,]  9628  8883 10524  6248   823   788    75   107   147   210   554
 [9,]  8228  5160 15380  3869  2571   672    41   342   118   254   443
[10,]  7488  2851 14529  2162  3256   684   189   272    75   248   304
[11,]  8514  8794 11274  1924  2503   335   199   291   157   213   182
[12,]  4256 12668 15110  1425  2222   199   134   169   257   187   487
[13,]  2387 10930  6970  5532  2174   686   158   145   227   131   597
[14,]  6942 11705  3342  7071  1182   655   143   249   167    82   525
[15,]  9209 11364 10542  5305   505   547    64   328   149   245   484
[16,]  8505  6838 14029  4819  2217   501    57   275   180   287   496
[17,]  7982  4021 12028  2217  2412   454   110   235   104   262   411
[18,]  8072 11311 10677   995  1872   224   133   240   169   308   256
[19,]  4060 15861  9710  3853  1576   113   196   204   271   308   668
[20,]  2555 13816 10141  5087  1639   508   165   118   225   205   470
[21,]  7185 14612 -2214  3723   726   476   135   227   193   138   731
[22,]  9143 14815  8205  3256   387   352    69   313   212   394   923
[23,]  9695  7885 10972  3238  1533   432    57   248   142   590   822
[24,]  8125  4571  8851  1540  1686   323   168   207    75   373   556
\end{Soutput}
\end{Schunk}
\end{table}

\begin{figure}[H]
\centering

\includegraphics[width=10cm]{cor_matrix.pdf}
\caption{Heatmap graf korelační matice}
\end{figure}

%% Kovariance
\section{Kovariance}
\subsection{Kovarianční matice}
\begin{table}[H]
\footnotesize
\begin{Schunk}
\begin{Soutput}
       V1                V2                V3                 V4         
 Min.   :-33.435   Min.   :-31.717   Min.   :-29.3043   Min.   :-17.957  
 1st Qu.:-16.168   1st Qu.:-21.842   1st Qu.:-13.8859   1st Qu.: -8.467  
 Median : 12.293   Median : -5.761   Median : -3.1087   Median :  2.473  
 Mean   :  5.465   Mean   : -2.097   Mean   :  0.1916   Mean   :  3.088  
 3rd Qu.: 28.946   3rd Qu.: 12.046   3rd Qu.: 13.0217   3rd Qu.:  9.364  
 Max.   : 50.000   Max.   : 49.978   Max.   : 49.9565   Max.   : 49.870  
       V5                V6                V7                 V8         
 Min.   :-23.435   Min.   :-34.304   Min.   :-29.1739   Min.   :-33.217  
 1st Qu.:-16.016   1st Qu.:-13.087   1st Qu.:-10.8207   1st Qu.:-17.967  
 Median :  6.620   Median :  4.533   Median :  0.5435   Median :  8.315  
 Mean   :  4.815   Mean   :  3.619   Mean   :  2.1472   Mean   :  5.156  
 3rd Qu.: 20.663   3rd Qu.: 14.897   3rd Qu.: 14.7663   3rd Qu.: 25.891  
 Max.   : 49.978   Max.   : 50.000   Max.   : 49.9565   Max.   : 50.000  
       V9               V10               V11               V12         
 Min.   :-34.435   Min.   :-36.435   Min.   :-33.435   Min.   :-31.174  
 1st Qu.:-10.658   1st Qu.:-17.337   1st Qu.:-23.321   1st Qu.:-14.668  
 Median :  8.826   Median :  6.196   Median : -1.522   Median : 10.457  
 Mean   :  4.632   Mean   :  3.081   Mean   : -1.861   Mean   :  4.583  
 3rd Qu.: 16.087   3rd Qu.: 23.522   3rd Qu.:  9.120   3rd Qu.: 16.321  
 Max.   : 50.000   Max.   : 50.000   Max.   : 50.000   Max.   : 50.000  
      V13               V14                V15                V16         
 Min.   :-35.739   Min.   :-25.9348   Min.   :-30.8696   Min.   :-33.783  
 1st Qu.: -8.989   1st Qu.:-17.0652   1st Qu.:-17.2446   1st Qu.:-17.913  
 Median :  3.685   Median : -3.0435   Median :  0.4348   Median : -5.250  
 Mean   :  2.385   Mean   :  0.5525   Mean   :  2.7237   Mean   : -2.403  
 3rd Qu.: 17.924   3rd Qu.: 14.1957   3rd Qu.: 21.7337   3rd Qu.:  8.435  
 Max.   : 50.000   Max.   : 50.0000   Max.   : 50.0000   Max.   : 50.000  
      V17                V18               V19               V20         
 Min.   :-34.4348   Min.   :-36.435   Min.   :-35.739   Min.   :-27.087  
 1st Qu.:-17.0543   1st Qu.:-19.696   1st Qu.:-19.120   1st Qu.:-19.397  
 Median : -0.6956   Median : -5.652   Median : -3.815   Median : -2.185  
 Mean   : -1.3895   Mean   : -3.584   Mean   : -3.570   Mean   : -2.415  
 3rd Qu.:  9.1413   3rd Qu.:  6.598   3rd Qu.:  8.679   3rd Qu.:  8.489  
 Max.   : 50.0000   Max.   : 50.000   Max.   : 50.000   Max.   : 49.978  
      V21               V22               V23               V24         
 Min.   :-29.304   Min.   :-29.174   Min.   :-25.565   Min.   :-33.783  
 1st Qu.: -9.234   1st Qu.:-14.495   1st Qu.:-12.223   1st Qu.: -9.245  
 Median :  3.457   Median : -5.609   Median :  6.832   Median :  3.538  
 Mean   :  2.636   Mean   :  1.337   Mean   :  4.915   Mean   :  3.788  
 3rd Qu.: 15.620   3rd Qu.: 16.364   3rd Qu.: 14.799   3rd Qu.: 17.571  
 Max.   : 50.000   Max.   : 49.978   Max.   : 49.978   Max.   : 49.978  
\end{Soutput}
\end{Schunk}
\end{table}

\begin{figure}[H]
\centering

\includegraphics[width=10cm]{cov_matrix_graph_heatmap.pdf}
\caption{Heatmap graf kovarianční matice}

\includegraphics[width=10cm]{cov_matrix_grap.pdf}
\caption{Graf kovarianční matice}
\end{figure}

\begin{figure}[H]
\centering
\includegraphics[width=10cm]{cov_matrix_graph_ggqqplot}
\caption{GGQQPlot graf korelační matice}

\end{figure}

%% Testování v kontingenčních tabulkách
\section{Testování v kontingenčních tabulkách}
\subsection{Pearsonův Chí-kvadrát test}

Následující chí-kvadrát test zkoumá zda má veličina nových případů v ČR stejné
rozdělení jako veličina nových případů v ČR. Vzhledem ke skutečnosti, že p-value je vyšší
než hladina významnosti ($\alpha$ = 0,05), tuto hypotézu nemůžeme zamítnout.

\begin{Schunk}
\begin{Soutput}
	Pearson's Chi-squared test

data:  new_tests_czechia and new_cases_czechia
X-squared = 147356, df = 146982, p-value = 0.245
\end{Soutput}
\end{Schunk}

%% Regrese
\section{Regrese}
\subsection{Lineární regrese}

Následující graf zobrazuje jakých hodnot bude s 95\% pravděpodobností nabývat
hodnota pacientů na ICU na milión v ČR v závislosti na počtu nově hospitalizovaných
pacientů na milión v ČR. Tato závislost je zde vyjádřena jako lineární funkce $y = -0.7746 + 0.1826x$.

\begin{figure}[H]
\centering
\includegraphics[width=10cm]{linear_regresion_graph.pdf}
\caption{Graf lineární regrese}

\end{figure}

\clearpage

\subsection{Kvadratická regrese}

Následující graf zobrazuje jakých hodnot bude s 95\% pravděpodobností nabývat
hodnota pacientů na ICU na milión v ČR v závislosti na počtu nově hospitalizovaných
pacientů na milión v ČR. Tato závislost je zde vyjádřena jako kvadratická funkce $y = 1.9982244 + 0.1074610x + 0.0001102x^2$.

\begin{figure}[H]
\centering
\includegraphics[width=10cm]{quadratic_regresion_graph.pdf}
\caption{Graf kvadraditcké regrese}

\end{figure}

\clearpage

%%%%%%%%%%%%%%%%%%%%%%%%%%%%%%%%%%%%%%%%%%%%%%%%%%%%%%%%%%%%
% Závěr
%%%%%%%%%%%%%%%%%%%%%%%%%%%%%%%%%%%%%%%%%%%%%%%%%%%%%%%%%%%%

\clearpage \phantomsection \addcontentsline{toc}{section}{Závěr}
\section*{Závěr}

V této semestrální práci byl analyzován vývoj epidemie nemoci Covid-19 v ČR.
Jak je zřejmé z grafů, zkoumaná data se neřídí dle normálního rozdělení pravděpodobnosti
a zpravidla jsou výrazně zešikmena vlevo. Pomocí grafů byly porovnány sedmidenní klouzavé
průměry nových případů na milión v Rakousku a České republice. Pomocí grafů byly
také vizualizovány další veličiny jako například počet nových hospitalizací, počet nových testů,
reprodukční číslo, počet pacientů na ICU či pozitivita testů. Pomocí testů bylo otestováno například
zda se střední hodnota nových případů rovná konkrétní hodnotě či zda se střední hodnota
nových případů výrazněji v průběhu času změnila. Tyto analýzy byly proté vzhledem k sešikmení dat
provedeny kromě t-testu také pomocí Wilcox testu, jelikož by jeho výsledky zešikmení dat nemělo
případně tolik ovlivnit. Pro řádné srovnání nových případů na milión je poté na data aplikován
test ANOVA, který tuto veličinu porovnává mezi ČR, Německem, Slovenskem, Polskem a Rakouskem.
Nakonec je pomocí regrese navržena lineární a kvadratická funkce popisující možnou závislost
přírůstku nových pacientů na ICU na milión na přírustku nově hospitalizovaných pacientů na milión.
Těmito operacemi práce jistě poskytuje bližší pohled na vývoj současné epidemie v naší zemi
jakož i nebezpečí, které tento virus představuje.

%%%%%%%%%%%%%%%%%%%%%%%%%%%%%%%%%%%%%%%%%%%%%%%%%%%%%%%%%%%%
% Použitá literatura
%%%%%%%%%%%%%%%%%%%%%%%%%%%%%%%%%%%%%%%%%%%%%%%%%%%%%%%%%%%%

\clearpage \phantomsection \addcontentsline{toc}{section}{\refname}

\begin{thebibliography}{99}	% parametr určuje nejširší položku

% nezlomitelné spojovníky lze zapisovat zkratkou "- nebo příkazem \babelhyphen{nobreak}

\bibitem{1}
Our World in Data
\textit{Data on COVID-19 (coronavirus)} [online]. 2021 [cit. 2021-11-18]. Dostupné z:~\url{https://github.com/owid/covid-19-data/tree/master/public/data}

\end{thebibliography}

%%%%%%%%%%%%%%%%%%%%%%%%%%%%%%%%%%%%%%%%%%%%%%%%%%%%%%%%%%%%
% Přílohy
%%%%%%%%%%%%%%%%%%%%%%%%%%%%%%%%%%%%%%%%%%%%%%%%%%%%%%%%%%%%

\clearpage \phantomsection \addcontentsline{toc}{section}{Seznam příloh}
\section*{Seznam příloh}

\noindent Příloha~A \dotfill \pageref{1}

% PŘÍLOHA A

\clearpage \phantomsection\label{prilohaA} \addcontentsline{toc}{section}{Příloha~A}
\section*{Příloha~A}

Příloha A zahrnuje ZIP soubor, který obsahuje: 

\begin{itemize}
    \item Zdrojové kódy
    \item Zdrojová data použitá v práci
\end{itemize}

%%%%%%%%%%%%%%%%%%%%%%%%%%%%%%%%%%%%%%%%%%%%%%%%%%%%%%%%%%%%
% Konec dokumentu
%%%%%%%%%%%%%%%%%%%%%%%%%%%%%%%%%%%%%%%%%%%%%%%%%%%%%%%%%%%%

\end{document}

% vim:sw=8:ts=8
% EOF
