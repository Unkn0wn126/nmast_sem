% VYSOKOŠKOLSKÁ SEMESTRÁLNÍ PRÁCE
% autor: Lukáš Milar, Tomáš Prudký
% název: Zpracování dat pro předmět NMAST
% !TEX encoding = UTF-8 Unicode
\documentclass[a4paper, 12pt]{article}

\usepackage{cmap}
\usepackage[utf8]{inputenc}
\usepackage[T1]{fontenc}
\usepackage[czech]{babel}
\selectlanguage{czech}
\usepackage{vskpupa}
\usepackage{subfig}
\usepackage{longtable}

%%%%%%%%%%%%%%%%%%%%%%%%%%%%%%%%%%%%%%%%%%%%%%%%%%%%%%%%%%%%
% Údaje o práci
%%%%%%%%%%%%%%%%%%%%%%%%%%%%%%%%%%%%%%%%%%%%%%%%%%%%%%%%%%%%

\def\jmenoFakulty{Fakulta elektrotechniky a informatiky}
\def\jmenoAutora{Bc. Lukáš Milar, Bc. Tomáš Prudký}
\def\nazevPrace{Zpracování dat pro předmět NMAST}
\def\typPrace{Semestrální práce}
\def\rok{2021}

%%%%%%%%%%%%%%%%%%%%%%%%%%%%%%%%%%%%%%%%%%%%%%%%%%%%%%%%%%%%
% Začátek dokumentu
%%%%%%%%%%%%%%%%%%%%%%%%%%%%%%%%%%%%%%%%%%%%%%%%%%%%%%%%%%%%

\usepackage{Sweave}
\begin{document}
\Sconcordance{concordance:semestralni_prace.tex:semestralni_prace.Rnw:%
1 28 1 1 0 20 1 1 17 6 1 1 11 19 1 1 7 6 1 1 130 20 0 1 7 20 0 1 7 20 0 %
1 2 14 1 1 5 2 1 1 5 13 1 1 5 2 1 1 5 12 1 1 5 2 1 1 5 13 1 1 5 2 1 1 5 %
11 1 1 10 2 1 1 10 12 1 1 10 2 1 1 11 6 1 1 20 2 1 1 20 8 1 1 5 2 1 1 5 %
6 1 1 5 2 1 1 5 6 1 1 5 2 1 1 5 6 1 1 5 2 1 1 5 6 1 1 5 2 1 1 5 7 1 1 5 %
2 1 1 5 6 1 1 5 7 1 1 5 2 1 1 5 5 1 1 5 2 1 1 5 5 1 1 5 8 1 1 7 10 1 1 %
23 20 0 1 2 12 1 1 5 10 1 1 5 8 1 1 8 3 1 1 2 13 0 2 2 13 0 2 2 13 0 1 %
2 2 1 1 2 13 0 2 2 13 0 2 2 13 0 1 2 3 1 1 2 13 0 1 2 6 1 1 2 13 0 2 2 %
13 0 2 2 13 0 1 2 2 1 1 2 13 0 2 2 13 0 1 2 2 1 1 2 8 0 2 2 8 0 1 2 4 1 %
1 2 13 0 1 2 2 1 1 2 7 0 2 2 7 0 1 2 3 1 1 3 10 0 1 2 3 1 1 5 7 1 1 5 2 %
1 1 5 7 1 1 5 2 1 1 5 7 1 1 5 7 1 1 3 5 0 1 2 5 1 1 6 127 0 1 1 43 0 1 %
2 52 0 1 2 4 1 1 5 8 1 1 5 44 0 1 1 1 2 4 1 1 5 2 1 1 5 5 1 1 5 8 1 1 2 %
7 0 1 2 5 1 1 10 7 1 1 16 58 1}


%%%%%%%%%%%%%%%%%%%%%%%%%%%%%%%%%%%%%%%%%%%%%%%%%%%%%%%%%%%%
% Úvodní strany
%%%%%%%%%%%%%%%%%%%%%%%%%%%%%%%%%%%%%%%%%%%%%%%%%%%%%%%%%%%%

\titulniStrana
\generujObsah			% obsah
\generujSeznamObrazku		% seznam obrázků
\generujSeznamTabulek		% seznam tabulek

%%%%%%%%%%%%%%%%%%%%%%%%%%%%%%%%%%%%%%%%%%%%%%%%%%%%%%%%%%%%
% Úvod
%%%%%%%%%%%%%%%%%%%%%%%%%%%%%%%%%%%%%%%%%%%%%%%%%%%%%%%%%%%%


\clearpage
\pagestyle{plain}		% zapne číslování stránek (sazba zápatí)
\phantomsection \addcontentsline{toc}{section}{Úvod}
\section*{Úvod}
\label{uvod}

Tato semestrální práce se zabývá analýzou vývoje epidemie nemoci Covid-19 v ČR. Za tímto účelem
jsou srovnány přírůstky nových případů s našimi sousedy, efektivita testů při odhalování nových případů, úmrtnost nakažených, střední hodnota hospitalizovaných a vývoj střední hodnoty nových případů v čase. Dále je provedeno srovnání středních hodnot nových případů na milión s našimi sousedy a zkoumáno jaké rozdělení pravděpodobnosti data následují. Nakonec je pomocí regrese analyzováno
podle jaké funkce se řídí přírůstek nových padientů na ICU v závislosti na nových hospitalizacích. Použitá data čerpají ze zdroje~\cite{1}.

%% Načítání dat start



%% Načítání dat end

%%%%%%%%%%%%%%%%%%%%%%%%%%%%%%%%%%%%%%%%%%%%%%%%%%%%%%%%%%%%
% Kapitoly
%%%%%%%%%%%%%%%%%%%%%%%%%%%%%%%%%%%%%%%%%%%%%%%%%%%%%%%%%%%%

%% Popis dat
\section{Popis dat}
Data použitá v této práci se zabývají veličinami ohledně nemoci Covid-19 a pochází
od společnosti Our World in Data. Tato data jsou denně aktualizována a obsahují
například informace o očkování, testech, hospitalizacích, nových případech,
nových úmrtích či reprodukčním čísle. Veškeré hodnoty jsou pozorovány napříč
mnoha státy. Pro bližší popis těchto dat vizte zdroj~\cite{1}.

Covid-19 (též COVID-19; z anglického spojení coronavirus disease 2019, což česky znamená koronavirové onemocnění 2019; výslovnost: [kovid devatenáct]; podle ICD-11 označené XN109) je vysoce infekční onemocnění, které je způsobeno koronavirem SARS-CoV-2. První případ byl identifikován v čínském Wu-chanu v prosinci 2019. Od té doby se virus rozšířil po celém světě, což způsobilo přetrvávající pandemii.

Příznaky nemoci covid-19 jsou různé, od bezpříznakového stavu až po závažné onemocnění, ale často zahrnují horečku, kašel, únavu, dýchací potíže a ztrátu čichu a chuti. Příznaky začínají jeden až čtrnáct dní po vystavení viru. U přibližně jednoho z pěti infikovaných jedinců se neobjeví žádné příznaky. Zatímco většina lidí má mírné příznaky, u některých lidí se vyvine syndrom akutní dechové tísně. Tento syndrom může být přivoděn cytokinovými bouřemi, víceorgánovým selháním, septickým šokem a krevními sraženinami. Bylo pozorováno dlouhodobější poškození orgánů (zejména plic a srdce). Existuje obava z významného počtu pacientů, kteří se zotavili z akutní fáze onemocnění, ale nadále pociťují řadu následků – známých jako dlouhodobý covid-19 – i několik měsíců poté. Mezi tyto účinky patří silná únava, ztráta paměti a další kognitivní problémy, slabá horečka, svalová slabost a dušnost.

Virus, který způsobuje covid-19, se šíří hlavně vzdušným přenosem, když je infikovaná osoba v blízkém kontaktu s jinou osobou. Malé kapičky a aerosoly obsahující virus se mohou šířit z nosu a úst infikované osoby při dýchání, kašlání, kýchání, zpěvu nebo mluvení. Ostatní lidé se mohou nakazit, pokud se virus dostane do jejich úst, nosu nebo očí. Virus se může šířit také kontaminovaným povrchem, i když to není považováno za hlavní cestu přenosu. Přesná cesta přenosu je zřídkakdy přesvědčivě prokázána, ale k infekci dochází hlavně tehdy, když jsou lidé dostatečně blízko sebe. Virus se může šířit až dva dny předtím, než infikované osoby projeví příznaky, a od jedinců, kteří nikdy nepociťují příznaky. Lidé zůstávají infekční po dobu až deseti dnů při středně závažných případech a dva týdny ve vážných případech. Virus se šíří snadněji ve vnitřních prostorách a v davu. Pro diagnózu onemocnění byly vyvinuty různé testovací metody. Standardní diagnostickou metodou je reverzní transkripční polymerázová řetězová reakce v reálném čase (PCR test) výtěrem z nosohltanu.

Preventivní opatření zahrnují fyzický či společenský odstup, umístění ohrožených osob do karantény, větrání vnitřních prostor, zakrývání úst a nosu při kašli a kýchání, mytí rukou a udržování neumytých rukou pryč od obličeje. Aby se minimalizovalo riziko přenosu, bylo na veřejnosti doporučeno použití roušek, obličejových masek nebo jiného zakrytí dýchacích cest. Bylo vyvinuto několik vakcín proti covidu-19, načež většina států světa zahájila očkovací kampaně a samotné očkování, jehož rozsah je ovšem závislý na přístupnosti dostatečného množství vakcín.

Ačkoli probíhají práce na vývoji léků, které zpomalují a zastavují virus, primární léčba je v současnosti symptomatická. Zahrnuje léčbu příznaků, podpůrnou péči, izolaci a některá experimentální opatření.

Vzhledem k velkému množství dat jsou v této práci použity zpravidla údaje pro
Českou republiku, ze kterých je dále využit užší výčet dostupných veličin.

%% Popisná statistika
\section{Popisná statistika}

V tabulkách níže jsou zobrazeny hodnoty popisné statistiky pro veličiny nových případů,
7denního klouzavého průměru nových případů, nových případů na milion, 7denního klouzavého průměru nových případů na milion,
hospitalizovaných pacientů a hospitalizovaných pacientů na milion v České republice. Hodnoty 7denního klouzavého průměru
lépe zachycují tyto veličiny v rámci dlouhodobých trendů, jelikož je eliminováno zkreslení v podobě menšího počtu uskutečněných
testů například během víkendů.

% latex table generated in R 4.1.1 by xtable 1.8-4 package
% Tue Dec 14 23:21:46 2021
\begin{table}[ht]
\centering
\begin{tabular}{rrr}
  \hline
 & Nové případy & 7denní klouzavý průměr nových případů \\ 
  \hline
průměr & 2940.58 & 2936.89 \\ 
  modus & 75.00 & 57.57 \\ 
  medián & 416.00 & 422.29 \\ 
  max & 17773.00 & 12954.86 \\ 
  min & -2214.00 & 2.71 \\ 
  šikmost & 1.55 & 1.16 \\ 
  špičatost & 1.38 & -0.07 \\ 
  odchylka & 4277.10 & 3876.20 \\ 
  variance & 18293577.15 & 15024928.55 \\ 
   \hline
\end{tabular}
\caption{Části popisné statistiky aplikované na data nových případů a jejich 7denního klouzavého průměru v ČR od 7. 3. 2020} 
\label{table:popisStat}
\end{table}% latex table generated in R 4.1.1 by xtable 1.8-4 package
% Tue Dec 14 23:21:46 2021
\begin{table}[ht]
\centering
\begin{tabular}{rrr}
  \hline
 & Nové případy na milión & 7denní klouzavý průměr nových případů na milión \\ 
  \hline
průměr & 274.19 & 273.85 \\ 
  modus & 6.99 & 5.37 \\ 
  medián & 38.79 & 39.38 \\ 
  max & 1657.22 & 1207.96 \\ 
  min & -206.44 & 0.25 \\ 
  šikmost & 1.55 & 1.16 \\ 
  špičatost & 1.38 & -0.07 \\ 
  odchylka & 398.81 & 361.43 \\ 
  variance & 159052.40 & 130633.34 \\ 
   \hline
\end{tabular}
\caption{Části popisné statistiky aplikované na data nových případů na milión a jejich 7denního klouzavého průměru v ČR od 7. 3. 2020} 
\label{table:popisStat}
\end{table}% latex table generated in R 4.1.1 by xtable 1.8-4 package
% Tue Dec 14 23:21:46 2021
\begin{table}[ht]
\centering
\begin{tabular}{rrr}
  \hline
 & Hospitalizovaní pacienti & Hospitalizovaní pacienti na milión \\ 
  \hline
průměr & 2370.49 & 221.03 \\ 
  modus & 69.00 & 6.43 \\ 
  medián & 339.00 & 31.61 \\ 
  max & 9509.00 & 886.66 \\ 
  min & 0.00 & 0.00 \\ 
  šikmost & 0.86 & 0.86 \\ 
  špičatost & -0.86 & -0.86 \\ 
  odchylka & 2973.01 & 277.22 \\ 
  variance & 8838793.84 & 76848.36 \\ 
   \hline
\end{tabular}
\caption{Části popisné statistiky aplikované na data nových hospitalizací a nových hospitalizací na milión v ČR od 7. 3. 2020} 
\label{table:popisStat}
\end{table}


%% Základní grafy
\section{Základní grafy}
\subsection{Histogram}

Následující histogram zobrazuje četnost hodnot klouzavého průměru nových případů v ČR od 7. 3. 2020.
Vzhledem k očividnému zešikmení dat vlevo byla pro lepší přehlednost data zlogaritmována.

%% histogram start

\begin{figure}[H]
\centering
\subfloat[\centering Klouzavý průměr nových případů]
{\includegraphics[width=7cm]{histogram_new_cases_smoothed.pdf}}
\qquad
\subfloat[\centering Logaritmická transformace]
{\includegraphics[width=7cm]{histogram_new_cases_smoothed_log.pdf}}
\caption{Klouzavý průměr nových případů v ČR od 7. 3. 2020}
\end{figure}

\clearpage

Následující histogram zobrazuje četnost hodnot nových případů na milión obyvatel v ČR od 7. 3. 2020.
Vzhledem k očividnému zešikmení dat vlevo byla pro lepší přehlednost data opět zlogaritmována.

%% histogram start

\begin{figure}[H]
\centering
\subfloat[\centering Nové případy na milión]
{\includegraphics[width=7cm]{histogram_new_cases_per_million.pdf}}
\qquad
\subfloat[\centering Logaritmická transformace]
{\includegraphics[width=7cm]{histogram_new_cases_per_million_log.pdf}}
\caption{Nové případy na milión v ČR od 7. 3. 2020}
\end{figure}


Následující histogram zobrazuje četnost hodnot 7denního klouzavého průměru nových případů na milión obyvatel v ČR od 7. 3. 2020.
Vzhledem k očividnému zešikmení dat vlevo byla pro lepší přehlednost data opět zlogaritmována.

%% histogram start

\begin{figure}[H]
\centering
\subfloat[\centering Klouzavý průměr nových případů na milión]
{\includegraphics[width=7cm]{histogram_new_cases_smoothed_per_million.pdf}}
\qquad
\subfloat[\centering Logaritmická transformace]
{\includegraphics[width=7cm]{histogram_new_cases_smoothed_per_million_log.pdf}}
\caption{Klouzavý průměr nových případů na milión v ČR od 7. 3. 2020}
\end{figure}


Následující histogram zobrazuje četnost hodnot hospitalizovaných pacientů v ČR od 7. 3. 2020.
Vzhledem k očividnému zešikmení dat vlevo byla pro lepší přehlednost data opět zlogaritmována.


%% histogram start

\begin{figure}[H]
\centering
\subfloat[\centering Hospitalizovaní pacienti]
{\includegraphics[width=7cm]{histogram_hosp_patients.pdf}}
\qquad
\subfloat[\centering Logaritmická transformace]
{\includegraphics[width=7cm]{histogram_hosp_patients_log.pdf}}
\caption{Hospitalizovaní pacienti v ČR od 7. 3. 2020}
\end{figure}

Následující histogram zobrazuje četnost hodnot nově testovaných v ČR od 7. 3. 2020.
Vzhledem k očividnému zešikmení dat vlevo byla pro lepší přehlednost data opět zlogaritmována.

%% histogram start

\begin{figure}[H]
\centering
\subfloat[\centering Nově testovaní]
{\includegraphics[width=7cm]{histogram_new_tests.pdf}}
\qquad
\subfloat[\centering Logaritmická transformace]
{\includegraphics[width=7cm]{histogram_new_tests_log.pdf}}
\caption{Nově testovaní v ČR od 7. 3. 2020}
\end{figure}


Následující histogram zobrazuje četnost hodnot nových případů v ČR od 7. 3. 2020.
Vzhledem k očividnému zešikmení dat vlevo byla pro lepší přehlednost data opět zlogaritmována.

%% histogram start

\begin{figure}[H]
\centering
\subfloat[\centering Nové případy]
{\includegraphics[width=7cm]{histogram_new_cases.pdf}}
\qquad
\subfloat[\centering Logaritmická transformace]
{\includegraphics[width=7cm]{histogram_new_cases_log.pdf}}
\caption{Nové případy v ČR od 7. 3. 2020}
\end{figure}



Následující histogram zobrazuje srovnání četnosti hodnot klouzavého průměru nových případů v ČR a Rakousku od 7. 3. 2020.
Vzhledem k očividnému zešikmení dat vlevo byla pro lepší přehlednost data opět zlogaritmována.

%% histogram start

\begin{figure}[H]
\centering
\subfloat[\centering Nové případy na milión pro ČR a Rakousko]
{\includegraphics[width=7cm]{histogram_new_cases_cz_au}}
\qquad
\subfloat[\centering Logaritmická transformace]
{\includegraphics[width=7cm]{histogram_new_cases_cz_au_log.pdf}}
\caption{Klouzavý průměr nových případů na milión pro Česko a Rakousko od 7. 3. 2020}
\end{figure}


\subsection{Bodový graf}
\begin{figure}[H]
\centering

\includegraphics[width=10cm]{bodovy_nove_pripady_log.pdf}
\caption{Bodový graf zlogaritmovaných nových případů}

\includegraphics[width=10cm]{bodovy_nove_testy.pdf}
\caption{Bodový graf nových testů}

\end{figure}
\begin{figure}[H]
\centering

\includegraphics[width=10cm]{bodovy_reprodukcni_cislo.pdf}
\caption{Bodový graf reprodukčního čísla}

\includegraphics[width=10cm]{bodovy_icu_patients.pdf}
\caption{Bodový graf pacientů na icu}

\end{figure}
\begin{figure}[H]
\centering

\includegraphics[width=10cm]{bodovy_hosp_patients.pdf}
\caption{Bodový graf hospitalizovaných pacientů}

\includegraphics[width=10cm]{bodovy_weekly_icu_admissions.pdf}
\caption{Bodový graf týdenních přírůstků na icu}

\end{figure}
\begin{figure}[H]
\centering

\includegraphics[width=10cm]{bodovy_weekly_hosp_admissions.pdf}
\caption{Bodový graf týdenních hospitalizací}

\includegraphics[width=10cm]{bodovy_positive_rate.pdf}
\caption{Bodový graf pozitivity testů}

\end{figure}
\begin{figure}[H]
\centering

\includegraphics[width=10cm]{bodovy_new_vaccinations.pdf}
\caption{Bodový graf nových očkování}

\includegraphics[width=10cm]{bodovy_excess_mortality.pdf}
\caption{Bodový graf smrtnosti}

\end{figure}
\subsection{Boxplot}
\begin{figure}[H]
\centering

\includegraphics[width=10cm]{boxplot_new_cases_per_million.pdf}
\caption{Boxplot graf pro nové případy na milión}

\includegraphics[width=10cm]{boxplot_reproduction_rate.pdf}
\caption{Boxplot graf pro reprodukční číslo}
\end{figure}

\begin{figure}[H]
\centering

\includegraphics[width=10cm]{boxplot_log_new_deaths.pdf}
\caption{Boxplot graf pro zlogaritmované nové smrti}
\end{figure}

\subsection{3D graf}
\begin{figure}[H]
\centering

\includegraphics[width=10cm]{3D_graf_new_cases_new_tests_total_cases.pdf}
\caption{3D graf počtu případů a počtu testů}

\includegraphics[width=10cm]{3D_graf_log_new_cases_new_tests_total_cases.pdf}
\caption{3D graf zlogaritmovaných počtu případů a počtu testů}

\end{figure}
\begin{figure}[H]

\includegraphics[width=10cm]{3D_graf_new_cases_new_tests_new_vaccinations.pdf}
\caption{3D graf počtu případů a počtu nových očkování}

\includegraphics[width=10cm]{3D_graf_new_cases.pdf}
\caption{3D graf počtu nových případů}

\end{figure}
\begin{figure}[H]

\includegraphics[width=10cm]{3D_graf_reproduction_rate.pdf}
\caption{3D graf reprodukčního čísla}

\end{figure}

\subsection{Hexbin}
\begin{figure}[H]
\centering

\includegraphics[width=10cm]{hexbin_graf_log_new_cases_new_deaths.pdf}
\caption{Hexbin graf nových zlog. nových případů a nových úmrtí}

\end{figure}
\clearpage

\subsection{Chernoff faces}

Pomocí chernoff faces jsou vyobrazeny obličeje na základě hodnot pro Česko. Hodnoty, které jsou použity jsou
$new_cases$, $new_deaths$, $new_tests$, $new_vaccinations$, $icu_patiens$. Každý obličej reprezentuje jednu funkci, která
je aplikována na zvolené hodnoty. Z grafu je zřejmé zešikmení dat vzhledem k velkým rozdílům mezi hodnotami průměru
a mediánu zobrazovaných veličin.

% latex table generated in R 4.1.1 by xtable 1.8-4 package
% Tue Dec 14 23:21:46 2021
\begin{table}[ht]
\centering
\begin{tabular}{rrrrrr}
  \hline
 & new\_cases & new\_deaths & new\_tests & new\_vaccinations & icu\_patients \\ 
  \hline
prumer & 2940.58 & 54.33 & 94464.09 & 42220.25 & 424.51 \\ 
  sd & 4277.10 & 72.32 & 83099.10 & 35340.13 & 558.31 \\ 
  maximum & 17773.00 & 295.00 & 416333.00 & 121742.00 & 2062.00 \\ 
  minumum & -2214.00 & -6.00 & 4537.00 & 266.00 & 0.00 \\ 
  sikmost & 1.55 & 1.11 & 1.39 & 0.79 & 1.23 \\ 
  spicatost & 1.38 & -0.06 & 1.22 & -0.57 & 0.43 \\ 
  iqr & 4275.75 & 107.00 & 82303.50 & 49771.00 & 799.00 \\ 
  median & 416.00 & 7.00 & 65211.00 & 33315.00 & 74.00 \\ 
   \hline
\end{tabular}
\caption{Hodnoty dat znázorněných pomocí Chernoffových obličejů} 
\label{table:popisStat}
\end{table}
\begin{figure}[H]
\small
\centering

\begin{Schunk}
\begin{Soutput}
effect of variables:
 modified item       Var               
 "height of face   " "new_cases"       
 "width of face    " "new_deaths"      
 "structure of face" "new_tests"       
 "height of mouth  " "new_vaccinations"
 "width of mouth   " "icu_patients"    
 "smiling          " "new_cases"       
 "height of eyes   " "new_deaths"      
 "width of eyes    " "new_tests"       
 "height of hair   " "new_vaccinations"
 "width of hair   "  "icu_patients"    
 "style of hair   "  "new_cases"       
 "height of nose  "  "new_deaths"      
 "width of nose   "  "new_tests"       
 "width of ear    "  "new_vaccinations"
 "height of ear   "  "icu_patients"    
\end{Soutput}
\end{Schunk}
\caption{Legenda Chernoff faces grafu tabulky popisné statistiky}
\end{figure}

\begin{figure}[H]
\small
\centering

\includegraphics[width=10cm]{faces_graf_covid.pdf}
\caption{Chernoff faces graf tabulky popisné statistiky}
\end{figure}
\clearpage

\subsection{QQPlot}

Na základě následujícího QQPlot grafu je možné dojít k závěru, že počty nových
úmrtí a počty nových případů v ČR se řídí dle podobného rozdělení pravděpodobnosti.

\begin{figure}[H]
\centering

\includegraphics[width=10cm]{qqplot_graph_new_cases_new_deaths.pdf}
\caption{QQPlot graf nových případů a nových úmrtí}
\end{figure}
\clearpage

Na základě následujícího QQPlot grafu je možné dojít k závěru, že počty nových
případů a počty nových testů v ČR se řídí dle podobného rozdělení pravděpodobnosti.

\begin{figure}[H]
\centering

\includegraphics[width=10cm]{qqplot_graph_new_tests_new_cases.pdf}
\caption{QQPlot graf nových testů a nových případů}

\end{figure}
\clearpage

%% Testování statistických hypotéz
\section{Testování statistických hypotéz}

\subsection{Jednovýběrový Studentův test vůči střední hodnotě}
Následující test testuje zda se střední hodnota nových případů v ČR rovná
hodnotě 3300 s hladinou významnosti $\alpha$ = 0.05. Testová statistika nabývá hodnoty -2.0168 při 575 stupních volnosti. Vzhledem k tomu, že hodnota p-value je nižší než hladina významnosti, tuto hypotézu zamítáme ve prospěch hypotézy alternativní, tudíž že se střední hodnota
nových případů v ČR nerovná hodnotě 3300.

\begin{Schunk}
\begin{Soutput}
	One Sample t-test

data:  new_cases_czechia
t = -2.0168, df = 575, p-value = 0.04418
alternative hypothesis: true mean is not equal to 3300
95 percent confidence interval:
 2590.550 3290.603
sample estimates:
mean of x 
 2940.576 
\end{Soutput}
\end{Schunk}

\clearpage

Následující test testuje zda se střední hodnota 7denního klouzavého průměru nových případů v ČR rovná
hodnotě 3300 s hladinou významnosti $\alpha$ = 0.05. Testová statistika nabývá hodnoty -2.2482 při 575 stupních volnosti. Vzhledem k tomu, že hodnota p-value je nižší než hladina významnosti, tuto hypotézu zamítáme ve prospěch hypotézy alternativní, tudíž že se střední hodnota
nových případů v ČR nerovná hodnotě 3300.
\begin{Schunk}
\begin{Soutput}
	One Sample t-test

data:  new_cases_smoothed_czechia
t = -2.2482, df = 575, p-value = 0.02494
alternative hypothesis: true mean is not equal to 3300
95 percent confidence interval:
 2619.672 3254.109
sample estimates:
mean of x 
  2936.89 
\end{Soutput}
\end{Schunk}

Následující test testuje zda se střední hodnota nových případů na milión v ČR rovná
hodnotě 300 s hladinou významnosti $\alpha$ = 0.05. Testová statistika nabývá hodnoty -1.5531 při 575 stupních volnosti. Vzhledem k tomu, že hodnota p-value je vyšší než hladina významnosti, tuto hypotézu nemůžeme zamítnout ve prospěch hypotézy alternativní.
\begin{Schunk}
\begin{Soutput}
	One Sample t-test

data:  new_cases_per_million_czechia
t = -1.5531, df = 575, p-value = 0.1209
alternative hypothesis: true mean is not equal to 300
95 percent confidence interval:
 241.5531 306.8289
sample estimates:
mean of x 
  274.191 
\end{Soutput}
\end{Schunk}

\clearpage

Následující test testuje zda se střední hodnota 7denního klouzavého průměru nových případů na milión v ČR rovná
hodnotě 300 s hladinou významnosti $\alpha$ = 0.05. Testová statistika nabývá hodnoty -1.7366 při 575 stupních volnosti. Vzhledem k tomu, že hodnota p-value je vyšší než hladina významnosti, tuto hypotézu nemůžeme zamítnout ve prospěch hypotézy alternativní.
\begin{Schunk}
\begin{Soutput}
	One Sample t-test

data:  new_cases_smoothed_per_million_czechia
t = -1.7366, df = 575, p-value = 0.08299
alternative hypothesis: true mean is not equal to 300
95 percent confidence interval:
 244.2687 303.4260
sample estimates:
mean of x 
 273.8474 
\end{Soutput}
\end{Schunk}


Následující test testuje zda se střední hodnota nových hospitalizovaných pacientů v ČR rovná
hodnotě 2000 s hladinou významnosti $\alpha$ = 0.05. Testová statistika nabývá hodnoty 2.9726 při 568 stupních volnosti. Vzhledem k tomu, že hodnota p-value je nižší než hladina významnosti, tuto hypotézu zamítáme ve prospěch hypotézy alternativní, tudíž že se střední hodnota
nových hospitalizovaných pacientů v ČR nerovná hodnotě 2000.
\begin{Schunk}
\begin{Soutput}
	One Sample t-test

data:  hosp_patients_czechia
t = 2.9726, df = 568, p-value = 0.003078
alternative hypothesis: true mean is not equal to 2000
95 percent confidence interval:
 2125.690 2615.294
sample estimates:
mean of x 
 2370.492 
\end{Soutput}
\end{Schunk}

\clearpage

Následující test testuje zda se střední hodnota nových hospitalizovaných pacientů na milión v ČR rovná
hodnotě 200 s hladinou významnosti $\alpha$ = 0.05. Testová statistika nabývá hodnoty 1.8099 při 568 stupních volnosti. Vzhledem k tomu, že hodnota p-value je vyšší než hladina významnosti, tuto hypotézu nemůžeme zamítnout ve prospěch hypotézy alternativní.
\begin{Schunk}
\begin{Soutput}
	One Sample t-test

data:  hosp_patients_per_million_czechia
t = 1.8099, df = 568, p-value = 0.07083
alternative hypothesis: true mean is not equal to 200
95 percent confidence interval:
 198.2078 243.8604
sample estimates:
mean of x 
 221.0341 
\end{Soutput}
\end{Schunk}

\clearpage

\subsection{Dvouvýběrový Studentův test}

Následující dvouvýběrový t-test testuje hypotézu, že střední hodnota nových případů v
první části dat z ČR je rovna střední hodnotě v druhé části. Vzhledem ke skutečnosti, že p-value je menší
než hladina významnosti ($\alpha$ = 0,05), zamítáme tuto hypotézu ve prospěch alternativní.
Při této hladině významnosti tudíž můžeme tvrdit, že střední hodnota nových případů v první
části dat z ČR se nerovná střední hodnotě z druhé části.

\begin{Schunk}
\begin{Soutput}
	Welch Two Sample t-test

data:  new_cases_czechia_p1 and new_cases_czechia_p2
t = -4.518, df = 537.03, p-value = 7.683e-06
alternative hypothesis: true difference in means is not equal to 0
95 percent confidence interval:
 -2272.4359  -895.1752
sample estimates:
mean of x mean of y 
 2148.674  3732.479 
\end{Soutput}
\end{Schunk}

\clearpage

Následující dvouvýběrový t-test testuje hypotézu, že střední hodnota nových případů
v ČR je rovna střední hodnotě nových případů v Německu. Testová statistika nabývá hodnoty
-11,133 při 843,1 stupních volnosti. Vzhledem ke skutečnosti, že p-value je menší
než hladina významnosti ($\alpha$ = 0,05), zamítáme tuto hypotézu ve prospěch alternativní.
Při této hladině významnosti tudíž můžeme tvrdit, že střední hodnota nových případů v Německu
se nerovná střední hodnotě nových případů v ČR.
\begin{Schunk}
\begin{Soutput}
	Welch Two Sample t-test

data:  new_cases_czechia and new_cases_germany
t = -11.133, df = 843.1, p-value < 2.2e-16
alternative hypothesis: true difference in means is not equal to 0
95 percent confidence interval:
 -5240.394 -3669.509
sample estimates:
mean of x mean of y 
 2940.576  7395.528 
\end{Soutput}
\end{Schunk}

Následující dvouvýběrový t-test testuje hypotézu, že střední hodnota nových případů na milión
v ČR je rovna střední hodnotě nových případů na milión na Slovensku. Vzhledem ke skutečnosti, že p-value je menší
než hladina významnosti ($\alpha$ = 0,05), zamítáme tuto hypotézu ve prospěch alternativní.
Při této hladině významnosti tudíž můžeme tvrdit, že střední hodnota nových případů na milión na Slovensku
se nerovná střední hodnotě nových případů na milión v ČR.

\begin{Schunk}
\begin{Soutput}
	Welch Two Sample t-test

data:  new_cases_per_million_czechia and new_cases_per_million_slovakia
t = 7.7283, df = 817.84, p-value = 3.194e-14
alternative hypothesis: true difference in means is not equal to 0
95 percent confidence interval:
 105.8863 177.9857
sample estimates:
mean of x mean of y 
  274.191   132.255 
\end{Soutput}
\end{Schunk}

\clearpage

Následující dvouvýběrový t-test testuje hypotézu, že střední hodnota nových případů na milión
v ČR je rovna střední hodnotě nových případů na milión v Německu. Vzhledem ke skutečnosti, že p-value je menší
než hladina významnosti ($\alpha$ = 0,05), zamítáme tuto hypotézu ve prospěch alternativní.
Při této hladině významnosti tudíž můžeme tvrdit, že střední hodnota nových případů na milión v Německu
se nerovná střední hodnotě nových případů na milión v ČR.

\begin{Schunk}
\begin{Soutput}
	Two Sample t-test

data:  new_cases_per_million_czechia and new_cases_per_million_germany
t = 10.844, df = 1150, p-value < 2.2e-16
alternative hypothesis: true difference in means is not equal to 0
95 percent confidence interval:
 152.3817 219.7075
sample estimates:
mean of x mean of y 
274.19103  88.14644 
\end{Soutput}
\end{Schunk}

\clearpage

Následující dvouvýběrový t-test testuje hypotézu, že střední hodnota nových případů na milión
v ČR je rovna střední hodnotě nových případů na milión v Polsku. Vzhledem ke skutečnosti, že p-value je menší
než hladina významnosti ($\alpha$ = 0,05), zamítáme tuto hypotézu ve prospěch alternativní.
Při této hladině významnosti tudíž můžeme tvrdit, že střední hodnota nových případů na milión v Polsku
se nerovná střední hodnotě nových případů na milión v ČR.

\begin{Schunk}
\begin{Soutput}
	Two Sample t-test

data:  new_cases_per_million_czechia and new_cases_per_million_poland
t = 7.5404, df = 1150, p-value = 9.477e-14
alternative hypothesis: true difference in means is not equal to 0
95 percent confidence interval:
 103.9326 177.0431
sample estimates:
mean of x mean of y 
 274.1910  133.7032 
\end{Soutput}
\end{Schunk}

Následující dvouvýběrový t-test testuje hypotézu, že střední hodnota nových případů na milión
v ČR je rovna střední hodnotě nových případů na milión v Rakousku. Vzhledem ke skutečnosti, že p-value je menší
než hladina významnosti ($\alpha$ = 0,05), zamítáme tuto hypotézu ve prospěch alternativní.
Při této hladině významnosti tudíž můžeme tvrdit, že střední hodnota nových případů na milión v Rakousku
se nerovná střední hodnotě nových případů na milión v ČR.

\begin{Schunk}
\begin{Soutput}
	Two Sample t-test

data:  new_cases_per_million_czechia and new_cases_per_million_austria
t = 7.2242, df = 1150, p-value = 9.157e-13
alternative hypothesis: true difference in means is not equal to 0
95 percent confidence interval:
  95.01377 165.86690
sample estimates:
mean of x mean of y 
 274.1910  143.7507 
\end{Soutput}
\end{Schunk}

\subsection{Wilcox test}

Následující Wilcoxonův testuje hypotézu, že střední hodnota nových případů na milión
v ČR je rovna střední hodnotě nových případů na milión na Slovensku. Vzhledem ke skutečnosti, že p-value je menší
než hladina významnosti ($\alpha$ = 0,05), zamítáme tuto hypotézu ve prospěch alternativní.
Při této hladině významnosti tudíž můžeme tvrdit, že střední hodnota nových případů na milión na Slovensku
se nerovná střední hodnotě nových případů na milión v ČR. Vzhledem k zešikmení dat poskytuje tento test přesnější
výsledky oproti dvouvýběrovému t-testu.

\begin{Schunk}
\begin{Soutput}
	Wilcoxon rank sum test with continuity correction

data:  new_cases_per_million_czechia and new_cases_per_million_slovakia
W = 205293, p-value = 2.97e-12
alternative hypothesis: true location shift is not equal to 0
\end{Soutput}
\end{Schunk}

Následující Wilcoxonův testuje hypotézu, že střední hodnota nových případů na milión
v ČR je rovna střední hodnotě nových případů na milión v Německu. Vzhledem ke skutečnosti, že p-value je menší
než hladina významnosti ($\alpha$ = 0,05), zamítáme tuto hypotézu ve prospěch alternativní.
Při této hladině významnosti tudíž můžeme tvrdit, že střední hodnota nových případů na milión v Německu
se nerovná střední hodnotě nových případů na milión v ČR. Vzhledem k zešikmení dat poskytuje tento test přesnější
výsledky oproti dvouvýběrovému t-testu.

\begin{Schunk}
\begin{Soutput}
	Wilcoxon rank sum test with continuity correction

data:  new_cases_per_million_czechia and new_cases_per_million_germany
W = 188720, p-value = 5.261e-05
alternative hypothesis: true location shift is not equal to 0
\end{Soutput}
\end{Schunk}

\clearpage

\subsection{Fisherův test}

Následující Fisherův test zkoumá zda jsou rozptyly hodnot nových případů na milión
v ČR a na Slovensku stejné. Vzhledem ke skutečnosti, že p-value je menší
než hladina významnosti ($\alpha$ = 0,05), zamítáme tuto hypotézu ve prospěch alternativní,
tudíž že jsou rozptyly těchto dat různé.

\begin{Schunk}
\begin{Soutput}
	F test to compare two variances

data:  new_cases_per_million_czechia and new_cases_per_million_slovakia
F = 4.5141, num df = 575, denom df = 575, p-value < 2.2e-16
alternative hypothesis: true ratio of variances is not equal to 1
95 percent confidence interval:
 3.832723 5.316656
sample estimates:
ratio of variances 
          4.514119 
\end{Soutput}
\end{Schunk}

\subsection{Shapiro Wilk test}

Následující Shapiro Wilk test testuje zda je veličina nových případů v ČR nabývá
normálního rozdělení. Vzhledem ke skutečnosti, že p-value je menší
než hladina významnosti ($\alpha$ = 0,05), zamítáme tuto hypotézu ve prospěch alternativní,
tudíž že tato veličina nenabývá normálního rozdělení.

\begin{Schunk}
\begin{Soutput}
	Shapiro-Wilk normality test

data:  new_cases_czechia
W = 0.72003, p-value < 2.2e-16
\end{Soutput}
\end{Schunk}

\clearpage

Následující Shapiro Wilk test testuje zda je veličina nových testů v ČR nabývá
normálního rozdělení. Vzhledem ke skutečnosti, že p-value je menší
než hladina významnosti ($\alpha$ = 0,05), zamítáme tuto hypotézu ve prospěch alternativní,
tudíž že tato veličina nenabývá normálního rozdělení.

\begin{Schunk}
\begin{Soutput}
	Shapiro-Wilk normality test

data:  new_tests_czechia
W = 0.82915, p-value < 2.2e-16
\end{Soutput}
\end{Schunk}

%% ANOVA
\section{ANOVA}

\begin{Schunk}
\begin{Soutput}
                                Df   Sum Sq  Mean Sq F value   Pr(>F)    
new_cases_per_million_germany    1 32058325 32058325 1020.43  < 2e-16 ***
new_cases_per_million_slovakia   1 37842394 37842394 1204.54  < 2e-16 ***
new_cases_per_million_poland     1  2637470  2637470   83.95  < 2e-16 ***
new_cases_per_million_austria    1   978148   978148   31.14 3.72e-08 ***
Residuals                      571 17938795    31416                     
---
Signif. codes:  0 '***' 0.001 '**' 0.01 '*' 0.05 '.' 0.1 ' ' 1
\end{Soutput}
\end{Schunk}

\begin{figure}[H]
\centering

\includegraphics[width=10cm]{anova_graph_1.pdf}
\caption{Anova graf nových testů, případů a úmrtí}

\end{figure}

\begin{figure}[H]
\centering

\includegraphics[width=10cm]{anova_graph_2.pdf}
\caption{Anova graf nových testů, případů a úmrtí}

\includegraphics[width=10cm]{anova_graph_3.pdf}
\caption{Anova graf nových testů, případů a úmrtí}

\end{figure}

\begin{figure}[H]
\centering

\includegraphics[width=10cm]{anova_graph_4.pdf}
\caption{Anova graf nových testů, případů a úmrtí}

\includegraphics[width=10cm]{anova_graph_5.pdf}
\caption{Anova graf nových testů, případů a úmrtí}

\end{figure}

\begin{figure}[H]
\centering

\includegraphics[width=10cm]{anova_graph_6.pdf}
\caption{Anova graf nových testů, případů a úmrtí}

\end{figure}

\clearpage
%% Variance
\section{Variance}

Níže jsou popsány střední hodnoty kvadrátů odchylek od střední hodnoty nových testů
a nových případů v ČR.
\begin{Schunk}
\begin{Soutput}
     Min.   1st Qu.    Median      Mean   3rd Qu.      Max. 
-61749419 -61749419 -61749419 -61749419 -61749419 -61749419 
\end{Soutput}
\end{Schunk}

%% Korelace
\section{Korelace}
\subsection{Korelační matice}
\begin{Schunk}
\begin{Soutput}
             [,1]        [,2]        [,3]         [,4]        [,5]         [,6]
 [1,]  1.00000000 -0.54948723 -0.06330228 -0.253692480  0.47034101  0.494350433
 [2,] -0.54948723  1.00000000  0.10390139  0.304800900 -0.39360394 -0.432884095
 [3,] -0.06330228  0.10390139  1.00000000  0.275607246 -0.36319740 -0.356972846
 [4,] -0.25369248  0.30480090  0.27560725  1.000000000 -0.13231174 -0.093132516
 [5,]  0.47034101 -0.39360394 -0.36319740 -0.132311738  1.00000000  0.614784606
 [6,]  0.49435043 -0.43288409 -0.35697285 -0.093132516  0.61478461  1.000000000
 [7,]  0.13896782 -0.24782103 -0.24161919  0.353519423  0.11078661  0.122226465
 [8,]  0.91233626 -0.51795164  0.02909308 -0.148550657  0.46939039  0.628649280
 [9,]  0.38448342 -0.21217598 -0.47549035  0.171646550  0.39143921  0.502760435
[10,]  0.68502263 -0.63069457  0.17535665 -0.330560209  0.33462476  0.148609873
[11,] -0.64882654  0.67676868 -0.09582765  0.134138762 -0.38301108 -0.369705597
[12,]  0.48112726 -0.28106527 -0.03481675  0.371208006  0.27535388  0.142940292
[13,]  0.36370252 -0.10131812 -0.12177299 -0.227428260  0.25138402  0.567271627
[14,] -0.25061897  0.19107363 -0.16457797  0.300565332 -0.02590209 -0.298372912
[15,]  0.65224771 -0.47181602  0.39361671 -0.248687122  0.14513369  0.160025578
[16,] -0.60115609  0.36763122 -0.40667922  0.035650419 -0.11878008 -0.136299917
[17,] -0.27929283  0.19938197  0.55497164  0.232351502 -0.31316661 -0.537716357
[18,] -0.58263640  0.46392759 -0.24461296 -0.056007261 -0.20473288 -0.075240976
[19,] -0.57197373  0.42554401  0.47913549  0.406242617 -0.44004508 -0.589522947
[20,] -0.34878234  0.44526385  0.08116204 -0.007928781 -0.41037971  0.050826337
[21,]  0.29677449 -0.21285419 -0.58429260 -0.171106614  0.25634765  0.001672992
[22,]  0.24980279 -0.06935493  0.43528786 -0.104363550 -0.11317344  0.065020929
[23,]  0.27047145 -0.25948603 -0.41088871  0.102449282  0.37728356  0.356624344
[24,]  0.64438294 -0.40150453  0.23617268  0.067762686  0.16029020  0.040603940
             [,7]        [,8]        [,9]       [,10]        [,11]       [,12]
 [1,]  0.13896782  0.91233626  0.38448342  0.68502263 -0.648826545  0.48112726
 [2,] -0.24782103 -0.51795164 -0.21217598 -0.63069457  0.676768676 -0.28106527
 [3,] -0.24161919  0.02909308 -0.47549035  0.17535665 -0.095827647 -0.03481675
 [4,]  0.35351942 -0.14855066  0.17164655 -0.33056021  0.134138762  0.37120801
 [5,]  0.11078661  0.46939039  0.39143921  0.33462476 -0.383011082  0.27535388
 [6,]  0.12222647  0.62864928  0.50276043  0.14860987 -0.369705597  0.14294029
 [7,]  1.00000000  0.10588364  0.47183815 -0.12094563 -0.396003117  0.70243376
 [8,]  0.10588364  1.00000000  0.40770185  0.56084862 -0.613745392  0.37776913
 [9,]  0.47183815  0.40770185  1.00000000 -0.14958123 -0.130837957  0.42596685
[10,] -0.12094563  0.56084862 -0.14958123  1.00000000 -0.703393692  0.27306666
[11,] -0.39600312 -0.61374539 -0.13083796 -0.70339369  1.000000000 -0.52613516
[12,]  0.70243376  0.37776913  0.42596685  0.27306666 -0.526135161  1.00000000
[13,] -0.45255954  0.45063651  0.12427722  0.11789103 -0.023479719 -0.42757786
[14,]  0.46682606 -0.26358165  0.40503744 -0.39469955  0.202854241  0.27722660
[15,] -0.35783885  0.69041056 -0.20157288  0.81320174 -0.506722870  0.06291983
[16,] -0.21519710 -0.60441403  0.03139535 -0.54507460  0.739181725 -0.49050054
[17,] -0.02167874 -0.35129619 -0.64287605  0.14882860 -0.025697540  0.20143055
[18,] -0.29559534 -0.56474419 -0.09736262 -0.72478193  0.744146397 -0.59415141
[19,]  0.17203370 -0.58181795 -0.46322626 -0.38445043  0.285458484  0.05858271
[20,] -0.37702642 -0.18381080 -0.18632785 -0.51760988  0.518994291 -0.52891745
[21,]  0.28854548  0.15213681  0.42044943  0.14524456 -0.008954537  0.22913874
[22,] -0.65417123  0.32802492 -0.41109552  0.53362687 -0.128194939 -0.28596395
[23,]  0.30795289  0.22526131  0.82623540 -0.02252771  0.002852469  0.35803125
[24,]  0.22804531  0.52552584  0.26305110  0.62902881 -0.558655403  0.61550857
             [,13]       [,14]       [,15]        [,16]       [,17]       [,18]
 [1,]  0.363702516 -0.25061897  0.65224771 -0.601156092 -0.27929283 -0.58263640
 [2,] -0.101318122  0.19107363 -0.47181602  0.367631222  0.19938197  0.46392759
 [3,] -0.121772986 -0.16457797  0.39361671 -0.406679217  0.55497164 -0.24461296
 [4,] -0.227428260  0.30056533 -0.24868712  0.035650419  0.23235150 -0.05600726
 [5,]  0.251384018 -0.02590209  0.14513369 -0.118780078 -0.31316661 -0.20473288
 [6,]  0.567271627 -0.29837291  0.16002558 -0.136299917 -0.53771636 -0.07524098
 [7,] -0.452559538  0.46682606 -0.35783885 -0.215197096 -0.02167874 -0.29559534
 [8,]  0.450636511 -0.26358165  0.69041056 -0.604414029 -0.35129619 -0.56474419
 [9,]  0.124277219  0.40503744 -0.20157288  0.031395351 -0.64287605 -0.09736262
[10,]  0.117891027 -0.39469955  0.81320174 -0.545074603  0.14882860 -0.72478193
[11,] -0.023479719  0.20285424 -0.50672287  0.739181725 -0.02569754  0.74414640
[12,] -0.427577855  0.27722660  0.06291983 -0.490500543  0.20143055 -0.59415141
[13,]  1.000000000 -0.48370608  0.37235746  0.007854382 -0.59342008  0.15076511
[14,] -0.483706078  1.00000000 -0.43772261  0.227934789 -0.09355538 -0.11651835
[15,]  0.372357465 -0.43772261  1.00000000 -0.551198459  0.08898115 -0.64927987
[16,]  0.007854382  0.22793479 -0.55119846  1.000000000 -0.17511202  0.64418732
[17,] -0.593420081 -0.09355538  0.08898115 -0.175112017  1.00000000 -0.15175859
[18,]  0.150765110 -0.11651835 -0.64927987  0.644187321 -0.15175859  1.00000000
[19,] -0.697742234  0.24346443 -0.41535576  0.058533021  0.68958292  0.17216858
[20,]  0.299396639 -0.20106417 -0.16859112  0.136032790 -0.08775309  0.52510429
[21,] -0.016238034  0.40625378 -0.06460999  0.049283183 -0.47724091 -0.15852356
[22,]  0.409076559 -0.57036169  0.76873691 -0.251268110  0.16070128 -0.27278028
[23,] -0.052452643  0.40725390 -0.15049820  0.289472148 -0.39521422 -0.09733461
[24,] -0.204324511 -0.13841903  0.49181120 -0.501437272  0.22031395 -0.61874308
            [,19]        [,20]        [,21]       [,22]        [,23]
 [1,] -0.57197373 -0.348782340  0.296774495  0.24980279  0.270471448
 [2,]  0.42554401  0.445263852 -0.212854193 -0.06935493 -0.259486029
 [3,]  0.47913549  0.081162036 -0.584292598  0.43528786 -0.410888707
 [4,]  0.40624262 -0.007928781 -0.171106614 -0.10436355  0.102449282
 [5,] -0.44004508 -0.410379707  0.256347654 -0.11317344  0.377283561
 [6,] -0.58952295  0.050826337  0.001672992  0.06502093  0.356624344
 [7,]  0.17203370 -0.377026416  0.288545480 -0.65417123  0.307952892
 [8,] -0.58181795 -0.183810804  0.152136808  0.32802492  0.225261313
 [9,] -0.46322626 -0.186327852  0.420449433 -0.41109552  0.826235405
[10,] -0.38445043 -0.517609879  0.145244559  0.53362687 -0.022527712
[11,]  0.28545848  0.518994291 -0.008954537 -0.12819494  0.002852469
[12,]  0.05858271 -0.528917453  0.229138740 -0.28596395  0.358031250
[13,] -0.69774223  0.299396639 -0.016238034  0.40907656 -0.052452643
[14,]  0.24346443 -0.201064170  0.406253782 -0.57036169  0.407253903
[15,] -0.41535576 -0.168591119 -0.064609988  0.76873691 -0.150498202
[16,]  0.05853302  0.136032790  0.049283183 -0.25126811  0.289472148
[17,]  0.68958292 -0.087753087 -0.477240905  0.16070128 -0.395214219
[18,]  0.17216858  0.525104295 -0.158523556 -0.27278028 -0.097334613
[19,]  1.00000000  0.118347789 -0.397322928 -0.25397306 -0.325774330
[20,]  0.11834779  1.000000000 -0.387687664  0.24459112 -0.381138177
[21,] -0.39732293 -0.387687664  1.000000000 -0.32992909  0.385696405
[22,] -0.25397306  0.244591125 -0.329929089  1.00000000 -0.333322819
[23,] -0.32577433 -0.381138177  0.385696405 -0.33332282  1.000000000
[24,] -0.16097037 -0.538536252  0.033197305  0.22275144  0.359999086
            [,24]
 [1,]  0.64438294
 [2,] -0.40150453
 [3,]  0.23617268
 [4,]  0.06776269
 [5,]  0.16029020
 [6,]  0.04060394
 [7,]  0.22804531
 [8,]  0.52552584
 [9,]  0.26305110
[10,]  0.62902881
[11,] -0.55865540
[12,]  0.61550857
[13,] -0.20432451
[14,] -0.13841903
[15,]  0.49181120
[16,] -0.50143727
[17,]  0.22031395
[18,] -0.61874308
[19,] -0.16097037
[20,] -0.53853625
[21,]  0.03319730
[22,]  0.22275144
[23,]  0.35999909
[24,]  1.00000000
\end{Soutput}
\end{Schunk}

\begin{figure}[H]
\centering

\includegraphics[width=10cm]{cor_matrix_gaph_heatmap.pdf}
\caption{Heatmap graf korelační matice}
\end{figure}

%% Kovariance
\section{Kovariance}
\subsection{Kovarianční matice}
\begin{Schunk}
\begin{Soutput}
              [,1]         [,2]        [,3]         [,4]         [,5]
 [1,]   11655.5199   -5581.3007   -157.7681   -507.15036    3681.1304
 [2,]   -5581.3007    8851.6449    225.6667    530.99638   -2684.5652
 [3,]    -157.7681     225.6667    532.9275    117.81159    -607.8261
 [4,]    -507.1504     530.9964    117.8116    342.86775    -177.6087
 [5,]    3681.1304   -2684.5652   -607.8261   -177.60870    5255.3913
 [6,]    3620.6649   -2762.9384   -559.0580   -116.99094    3023.5217
 [7,]    1366.3623   -2123.4203   -507.9855    596.15942     731.4348
 [8,]   41641.5344  -20601.8949    283.9420  -1162.90399   14386.0870
 [9,]   49127.1286  -23625.7790 -12991.3333   3761.63225   33584.9565
[10,]  271050.3170 -217475.4601  14836.6377 -22433.29529   88907.7826
[11,] -275592.5308  250510.1051  -8703.5797   9772.16123 -109241.3043
[12,]  119305.1105  -60736.8732  -1846.1014  15787.51993   45848.6957
[13,]  205291.0054  -49837.5761 -14697.4783 -22017.37500   95279.1739
[14,]  -68113.3080   45254.8333  -9564.4058  14010.52536   -4727.0435
[15,]  269795.3587 -170075.1087  34814.7826 -17643.01087   40311.3043
[16,] -285503.4964  152153.7754 -41299.4638   2903.93116  -37879.5217
[17,]  -70292.1884   43729.9855  29866.6377  10029.75362  -52924.8261
[18,]  -58929.0399   40891.0362  -5290.2899   -971.56884  -13904.5217
[19,]  -21161.2754   13720.0725   3790.4638   2577.79710  -10932.0000
[20,]   -3419.7101    3804.5072    170.1594    -13.33333   -2701.8261
[21,]    2275.2772   -1422.1196   -957.8696   -224.99457    1319.6957
[22,]    1561.2409    -377.7428    581.7246   -111.87138    -474.9565
[23,]    3170.6467   -2650.8587  -1029.9565    205.98370    2969.8261
[24,]   12932.9565   -7022.4783   1013.5652    233.26087    2160.2174
               [,6]         [,7]         [,8]         [,9]         [,10]
 [1,]   3620.664855    1366.3623    41641.534    49127.129    271050.317
 [2,]  -2762.938406   -2123.4203   -20601.895   -23625.779   -217475.460
 [3,]   -559.057971    -507.9855      283.942   -12991.333     14836.638
 [4,]   -116.990942     596.1594    -1162.904     3761.632    -22433.295
 [5,]   3023.521739     731.4348    14386.087    33584.957     88907.783
 [6,]   4602.302536     755.1594    18030.259    40367.013     36949.998
 [7,]    755.159420    8294.1449     4076.812    50857.710    -40369.667
 [8,]  18030.259058    4076.8116   178735.955   203998.317    869022.042
 [9,]  40367.012681   50857.7101   203998.317  1400734.650   -648834.900
[10,]  36949.998188  -40369.6667   869022.042  -648834.900  13432546.563
[11,] -98677.153986 -141891.8406 -1020862.415  -609234.792 -10142631.676
[12,]  22272.835145  146934.7246   366830.705  1157942.502   2298693.705
[13,] 201203.918478 -215486.1304   996070.614   769001.614   2259007.745
[14,] -50956.452899  107026.9420  -280526.149  1206772.257  -3641648.453
[15,]  41594.228261 -124861.7391  1118328.707  -914041.946  11419144.315
[16,] -40676.307971  -86214.5507 -1124084.221   163456.286  -8788069.395
[17,] -85039.811594   -4602.5797  -346226.942 -1773726.449   1271589.797
[18,]  -4781.981884  -25220.2464  -223678.199  -107953.344  -2488584.895
[19,] -13705.289855    5369.0725   -84293.290  -187875.884   -482858.029
[20,]    313.144928   -3118.3623    -7057.420   -20027.406   -172286.290
[21,]      8.059783    1866.1304     4567.538    35337.321     37802.538
[22,]    255.356884   -3448.9275     8028.226   -28166.150    113220.139
[23,]   2626.994565    3045.3043    10340.777   106179.908     -8965.136
[24,]    512.086957    3860.9565    41303.609    57877.000    428585.696
              [,11]        [,12]        [,13]        [,14]       [,15]
 [1,]   -275592.531   119305.111   205291.005   -68113.308   269795.36
 [2,]    250510.105   -60736.873   -49837.576    45254.833  -170075.11
 [3,]     -8703.580    -1846.101   -14697.478    -9564.406    34814.78
 [4,]      9772.161    15787.520   -22017.375    14010.525   -17643.01
 [5,]   -109241.304    45848.696    95279.174    -4727.043    40311.30
 [6,]    -98677.154    22272.835   201203.918   -50956.453    41594.23
 [7,]   -141891.841   146934.725  -215486.130   107026.942  -124861.74
 [8,]  -1020862.415   366830.705   996070.614  -280526.149  1118328.71
 [9,]   -609234.792  1157942.502   769001.614  1206772.257  -914041.95
[10,] -10142631.676  2298693.705  2259007.745 -3641648.453 11419144.32
[11,]  15479111.042 -4754486.187  -482973.929  2009135.895 -7638350.77
[12,]  -4754486.187  5275535.042 -5134595.679  1602951.438   553702.99
[13,]   -482973.929 -5134595.679 27334779.853 -6366365.207  7458885.49
[14,]   2009135.895  1602951.438 -6366365.207  6337302.341 -4221893.59
[15,]  -7638350.772   553702.989  7458885.489 -4221893.587 14679530.63
[16,]  12793299.953 -4955997.764   180645.924  2524185.007 -9290143.72
[17,]   -235692.638  1078549.449 -7232717.261  -549038.029   794759.17
[18,]   2742820.062 -1278487.004   738455.902  -274797.297 -2330525.98
[19,]    384871.623    46110.797 -1250122.957   210033.058  -545351.17
[20,]    185440.493  -110329.116   142158.783   -45968.029   -58662.39
[21,]     -2501.832    37374.353    -6028.832    72625.902   -17579.14
[22,]    -29197.839   -38023.437   123813.842   -83120.736   170506.38
[23,]      1218.582    89292.462   -29777.332   111321.250   -62610.62
[24,]   -408606.217   262818.130  -198594.043   -64779.261   350301.87
             [,16]       [,17]         [,18]        [,19]         [,20]
 [1,]  -285503.496   -70292.19   -58929.0399   -21161.275   -3419.71014
 [2,]   152153.775    43729.99    40891.0362    13720.072    3804.50725
 [3,]   -41299.464    29866.64    -5290.2899     3790.464     170.15942
 [4,]     2903.931    10029.75     -971.5688     2577.797     -13.33333
 [5,]   -37879.522   -52924.83   -13904.5217   -10932.000   -2701.82609
 [6,]   -40676.308   -85039.81    -4781.9819   -13705.290     313.14493
 [7,]   -86214.551    -4602.58   -25220.2464     5369.072   -3118.36232
 [8,] -1124084.221  -346226.94  -223678.1993   -84293.290   -7057.42029
 [9,]   163456.286 -1773726.45  -107953.3442  -187875.884  -20027.40580
[10,] -8788069.395  1271589.80 -2488584.8949  -482858.029 -172286.28986
[11,] 12793299.953  -235692.64  2742820.0616   384871.623  185440.49275
[12,] -4955997.764  1078549.45 -1278487.0036    46110.797 -110329.11594
[13,]   180645.924 -7232717.26   738455.9022 -1250122.957  142158.78261
[14,]  2524185.007  -549038.03  -274797.2971   210033.058  -45968.02899
[15,] -9290143.717   794759.17 -2330525.9783  -545351.174  -58662.39130
[16,] 19351592.341 -1795790.88  2654828.3841    88238.725   54346.42029
[17,] -1795790.884  5434541.62  -331436.7536   550893.797  -18578.59420
[18,]  2654828.384  -331436.75   877671.2101    55273.855   44676.63768
[19,]    88238.725   550893.80    55273.8551   117435.623    3683.23188
[20,]    54346.420   -18578.59    44676.6377     3683.232    8247.79710
[21,]    15395.685   -79006.17   -10546.3370    -9669.087   -2500.30435
[22,]   -63988.591    21687.41   -14794.0036    -5038.420    1285.92754
[23,]   138269.293  -100040.22    -9901.3370   -12122.087   -3758.47826
[24,]  -410074.652    95479.78  -107761.6522   -10254.957   -9092.26087
              [,21]       [,22]        [,23]        [,24]
 [1,]   2275.277174   1561.2409    3170.6467   12932.9565
 [2,]  -1422.119565   -377.7428   -2650.8587   -7022.4783
 [3,]   -957.869565    581.7246   -1029.9565    1013.5652
 [4,]   -224.994565   -111.8714     205.9837     233.2609
 [5,]   1319.695652   -474.9565    2969.8261    2160.2174
 [6,]      8.059783    255.3569    2626.9946     512.0870
 [7,]   1866.130435  -3448.9275    3045.3043    3860.9565
 [8,]   4567.538043   8028.2264   10340.7772   41303.6087
 [9,]  35337.320652 -28166.1504  106179.9076   57877.0000
[10,]  37802.538043 113220.1395   -8965.1359  428585.6957
[11,]  -2501.831522 -29197.8388    1218.5815 -408606.2174
[12,]  37374.353261 -38023.4366   89292.4620  262818.1304
[13,]  -6028.831522 123813.8424  -29777.3315 -198594.0435
[14,]  72625.902174 -83120.7355  111321.2500  -64779.2609
[15,] -17579.141304 170506.3804  -62610.6196  350301.8696
[16,]  15395.684783 -63988.5906  138269.2935 -410074.6522
[17,] -79006.173913  21687.4058 -100040.2174   95479.7826
[18,] -10546.336957 -14794.0036   -9901.3370 -107761.6522
[19,]  -9669.086957  -5038.4203  -12122.0870  -10254.9565
[20,]  -2500.304348   1285.9275   -3758.4783   -9092.2609
[21,]   5042.940217  -1356.3424    2974.0489     438.2609
[22,]  -1356.342391   3351.3025   -2095.2337    2397.2609
[23,]   2974.048913  -2095.2337   11790.2011    7266.9130
[24,]    438.260870   2397.2609    7266.9130   34560.1739
\end{Soutput}
\end{Schunk}

\begin{figure}[H]
\centering

\includegraphics[width=10cm]{cov_matrix_graph_heatmap.pdf}
\caption{Heatmap graf kovarianční matice}

\includegraphics[width=10cm]{cov_matrix_grap.pdf}
\caption{Graf kovarianční matice}
\end{figure}

\begin{figure}[H]
\centering
\includegraphics[width=10cm]{cov_matrix_graph_ggqqplot}
\caption{GGQQPlot graf korelační matice}

\end{figure}

%% Testování v kontingenčních tabulkách
\section{Testování v kontingenčních tabulkách}
\subsection{Pearsonův Chí-kvadrát test}

Následující kontingenční tabulka zobrazuje četnost výskytu hodnot nových případů na milión
pro ČR, Slovensko, Polsko, Německo a Rakousko v rámci skupin. Pomocí chí-kvadrát testu,
který poté následuje zamítáme hypotézu, že se tato veličina napříč zmíněnými zeměmi
řídí stejným rozdělením pravděpodobnosti.

% latex table generated in R 4.1.1 by xtable 1.8-4 package
% Tue Dec 14 23:21:47 2021
\begin{table}[ht]
\centering
\begin{tabular}{rrrr}
  \hline
 & x$<$10 & 10-100 & 100$>$x \\ 
  \hline
Czech & 104 & 232 & 240 \\ 
  Slovakia & 219 & 149 & 208 \\ 
  Poland & 198 & 172 & 206 \\ 
  Germany & 122 & 272 & 181 \\ 
  Austria & 104 & 205 & 267 \\ 
   \hline
\end{tabular}
\caption{Kontingenční tabulka nových případů na milión} 
\label{table:popisStat}
\end{table}
\begin{Schunk}
\begin{Soutput}
	Pearson's Chi-squared test

data:  table
X-squared = 146.95, df = 8, p-value < 2.2e-16
\end{Soutput}
\end{Schunk}

Následující chí-kvadrát test zkoumá zda má veličina nových případů v ČR stejné
rozdělení jako veličina nových případů v ČR. Vzhledem ke skutečnosti, že p-value je vyšší
než hladina významnosti ($\alpha$ = 0,05), tuto hypotézu nemůžeme zamítnout.

\begin{Schunk}
\begin{Soutput}
	Pearson's Chi-squared test

data:  new_tests_czechia and new_cases_czechia
X-squared = 147356, df = 146982, p-value = 0.245
\end{Soutput}
\end{Schunk}

%% Regrese
\section{Regrese}
\subsection{Lineární regrese}

Následující graf zobrazuje jakých hodnot bude s 95\% pravděpodobností nabývat
hodnota pacientů na ICU na milión v ČR v závislosti na počtu nově hospitalizovaných
pacientů na milión v ČR. Tato závislost je zde vyjádřena jako lineární funkce $y = -0.7746 + 0.1826x$.

\begin{figure}[H]
\centering
\includegraphics[width=10cm]{linear_regresion_graph.pdf}
\caption{Graf lineární regrese}

\end{figure}

\clearpage

\subsection{Kvadratická regrese}

Následující graf zobrazuje jakých hodnot bude s 95\% pravděpodobností nabývat
hodnota pacientů na ICU na milión v ČR v závislosti na počtu nově hospitalizovaných
pacientů na milión v ČR. Tato závislost je zde vyjádřena jako kvadratická funkce $y = 1.9982244 + 0.1074610x + 0.0001102x^2$.

\begin{figure}[H]
\centering
\includegraphics[width=10cm]{quadratic_regresion_graph.pdf}
\caption{Graf kvadraditcké regrese}

\end{figure}

\clearpage

%%%%%%%%%%%%%%%%%%%%%%%%%%%%%%%%%%%%%%%%%%%%%%%%%%%%%%%%%%%%
% Závěr
%%%%%%%%%%%%%%%%%%%%%%%%%%%%%%%%%%%%%%%%%%%%%%%%%%%%%%%%%%%%

\clearpage \phantomsection \addcontentsline{toc}{section}{Závěr}
\section*{Závěr}

V této semestrální práci byl analyzován vývoj epidemie nemoci Covid-19 v ČR.
Jak je zřejmé z grafů, zkoumaná data se neřídí dle normálního rozdělení pravděpodobnosti
a zpravidla jsou výrazně zešikmena vlevo. Pomocí grafů byly porovnány sedmidenní klouzavé
průměry nových případů na milión v Rakousku a České republice. Pomocí grafů byly
také vizualizovány další veličiny jako například počet nových hospitalizací, počet nových testů,
reprodukční číslo, počet pacientů na ICU či pozitivita testů. Pomocí testů bylo otestováno například
zda se střední hodnota nových případů rovná konkrétní hodnotě či zda se střední hodnota
nových případů výrazněji v průběhu času změnila. Tyto analýzy byly proté vzhledem k sešikmení dat
provedeny kromě t-testu také pomocí Wilcox testu, jelikož by jeho výsledky zešikmení dat nemělo
případně tolik ovlivnit. Pro řádné srovnání nových případů na milión je poté na data aplikován
test ANOVA, který tuto veličinu porovnává mezi ČR, Německem, Slovenskem, Polskem a Rakouskem.
Nakonec je pomocí regrese navržena lineární a kvadratická funkce popisující možnou závislost
přírůstku nových pacientů na ICU na milión na přírustku nově hospitalizovaných pacientů na milión.
Těmito operacemi práce jistě poskytuje bližší pohled na vývoj současné epidemie v naší zemi
jakož i nebezpečí, které tento virus představuje.

%%%%%%%%%%%%%%%%%%%%%%%%%%%%%%%%%%%%%%%%%%%%%%%%%%%%%%%%%%%%
% Použitá literatura
%%%%%%%%%%%%%%%%%%%%%%%%%%%%%%%%%%%%%%%%%%%%%%%%%%%%%%%%%%%%

\clearpage \phantomsection \addcontentsline{toc}{section}{\refname}

\begin{thebibliography}{99}	% parametr určuje nejširší položku

% nezlomitelné spojovníky lze zapisovat zkratkou "- nebo příkazem \babelhyphen{nobreak}

\bibitem{1}
Our World in Data
\textit{Data on COVID-19 (coronavirus)} [online]. 2021 [cit. 2021-11-18]. Dostupné z:~\url{https://github.com/owid/covid-19-data/tree/master/public/data}

\end{thebibliography}

%%%%%%%%%%%%%%%%%%%%%%%%%%%%%%%%%%%%%%%%%%%%%%%%%%%%%%%%%%%%
% Přílohy
%%%%%%%%%%%%%%%%%%%%%%%%%%%%%%%%%%%%%%%%%%%%%%%%%%%%%%%%%%%%

\clearpage \phantomsection \addcontentsline{toc}{section}{Seznam příloh}
\section*{Seznam příloh}

\noindent Příloha~A \dotfill \pageref{1}

% PŘÍLOHA A

\clearpage \phantomsection\label{prilohaA} \addcontentsline{toc}{section}{Příloha~A}
\section*{Příloha~A}

Příloha A zahrnuje ZIP soubor, který obsahuje: 

\begin{itemize}
    \item Zdrojové kódy
    \item Zdrojová data použitá v práci
\end{itemize}

%%%%%%%%%%%%%%%%%%%%%%%%%%%%%%%%%%%%%%%%%%%%%%%%%%%%%%%%%%%%
% Konec dokumentu
%%%%%%%%%%%%%%%%%%%%%%%%%%%%%%%%%%%%%%%%%%%%%%%%%%%%%%%%%%%%

\end{document}

% vim:sw=8:ts=8
% EOF
